% --------------------------------------------------
% DOCUMENT CLASS
% --------------------------------------------------

\documentclass[../quali_slides.tex]{subfiles}

\begin{document}
	
\section{Introduction}

% --------------------------------------------------
% SLIDE
% --------------------------------------------------

\begin{frame}{Introduction}
	
	Brazilian regions have heterogeneous economic matrices that respond in diverse ways to the decisions of the monetary authority. \cite{bertanha_efeitos_2008}.		
	
\end{frame}

% --------------------------------------------------
% SLIDE
% --------------------------------------------------

\begin{frame}{Introduction}

	Objectives:
	
	\begin{itemize}
		
		\item Develop a NK DSGE model with:
			
			\begin{itemize}
				\item two regions with distinct structures of production;
				\item monetary-policy shocks.
			\end{itemize}
		
		\item Demonstrate that different regions react in distinct ways to the monetary policy.
		
	\end{itemize}

\end{frame}

% --------------------------------------------------
% SLIDE
% --------------------------------------------------

\begin{frame}{What is a NK DSGE model?}
	
	NK DSGE model is a macroeconomic tool with:
	
	\begin{itemize}
		
		\item \textbf{New Keynesian}: monopolistic competition, nominal rigidities, short-run non-neutrality of monetary policy.
		
		\item \textbf{Dynamic}: shows the changes over time.
		
		\item \textbf{Stochastic}: considers random and uncertainty.
		
		\item \textbf{General Equilibrium}: agents optimize and markets clear (microfoundations).
		
	\end{itemize}
	
\end{frame}

% --------------------------------------------------
% RASCUNHO
% --------------------------------------------------

\begin{comment}
	
	\item A modelagem macroeconômica é uma importante ferramenta para estudar as ligações entre a economia monetária e os resultados dos agregados de um país, \textcite{gali_monetary_2015}.
	
	\item \textit{Na realidade, a maior parte das tolices já escritas e que se continuam a escrever sobre economia poderia ter sido poupada se todo economista fosse obrigado a expor suas ideias construindo um modelo matemático} --- \textcite[p.68]{simonsen_microeconomia_1979}.
	
	\item Os modelos DSGE começaram a ser usados para estruturar a Teoria dos Ciclos Reais de Negócios (\textit{Real Business Cycle Theory, RBC}), com os trabalhos seminais de \textcite{kydland_time_1982} e \textcite{prescott_theory_1986}, \textcite{gali_monetary_2015}.

	\item As principais características dos modelos RBC são: eficiência dos ciclos de negócios; importância dos choques de tecnologia como fontes de flutuações; papel limitado dos fatores monetários.

	\item Em paralelo aos modelos RBC, surgiram os modelos Novos Keynesianos (\textit{New Keynesian Theory, NK}), que procuram dar microfundamentos aos conceitos Keynesianos, \textcite[p.26]{gali_macroeconomic_2007}.
	
	\item As características de destaque dos modelos NK são: competição monopolística; rigidez nominal de preços; não-neutralidade da moeda no curto prazo.	
	
\end{comment}


\end{document}