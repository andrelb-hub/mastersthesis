% --------------------------------------------------
% DOCUMENT CLASS
% --------------------------------------------------

\documentclass[../thesis.tex]{subfiles}

\begin{document}

	\newpage

	\section{Results}\label{sec:results}	

	In this section, the Impulse Response Functions of each variable of the model is presented and their reaction to the monetary policy shock is discussed.
	
	%Figures \eqref{fig:ZMt-irf} and \eqref{fig:ZMt-irf2} shows a monetary policy shock of $1\%$, which will create a reaction in the economy starting by raising the interest rate by the same amount. The new interest rate, which is also the price of the capital rent, will make household review their consumption decision, as now it is more advantageous to save now (through investment) and consume more in the future. This decision will raise the investment and lower demand. On the other hand, although capital rent is higher, supply of capital and demand for more production is also higher, making firms decide to produce more. Although firms now demand more labor, household will supply more labor, because they want a higher income to purchase more investments. As wages are falling, it is fair to assume the supply is higher than demand. Wages will raise, but less than the price of capital. As households demand more product to transform in investments, the production will raise. The capital rent is higher, but wages are lower, making the marginal cost fall. Both consumer price and firm price levels will first raise and then fall, as the movement started by the monetary shock walks to stability of the economy. In the new equilibrium, there is a fall in the nominal levels, indicating a successful monetary shock. Investment, Capital, Labor and Production are now in a higher level, while Consumption is in a lower level, also indicating a positive result of the monetary policy shock.
	
	Figures \eqref{fig:ZMt-irf} and \eqref{fig:ZMt-irf2} depict the reaction of each variable of the model to a $1\%$ monetary policy shock, initiating a response in the economy by elevating the interest rate by the same percentage. The revised interest rate, also representing the price of capital rent, prompts households to reevaluate their consumption decisions. It becomes more advantageous to save now through investment and consume more in the future. This choice leads to an increase in investment and a decrease in demand.
	
	However, despite the higher capital rent, firms to decide to produce more, driven by a higher demand for final products (which will become investment for the next period) and a higher supply of capital. While firms now require more labor, households also supply more labor, seeking higher income for additional investments. As wages decrease, it is reasonable to assume that supply exceeds demand. Wages will rise, but less than the price of capital. As households demand more production to transform into investments, production increases. Although capital rent is higher, lower wages lead to a decline in marginal costs.
	
	Consumer and firm price levels initially rise and then fall as the effects of the monetary shock progress towards stabilizing the economy. In the new equilibrium, there is a decrease in nominal levels, indicating a successful monetary shock. Investment, capital, labor, and production are now at higher levels, while consumption is at a lower level, reflecting a positive outcome of the monetary policy shock. The outcome described can be synthesized in the flow on \eqref{eq:flow}. It is important to note that price variations are small in magnitude due to the fact that the capital price corresponds to the interest rate of the economy, denoted as $R \in (0,1)$. %It is important to point that prices variation are small in magnitude because of the fact that the capital price is the interest rate of the economy, and as such it is $R \in (0,1)$.	
	\begin{align}
		\begin{matrix}
			\hat{Z}_{M} \uparrow &\implies &\hat{R}_{}\uparrow &\implies &\hat{I}_{\eta}^D \uparrow &\implies &\hat{C}_{\eta} \downarrow &\implies \\
			\hat{K}_{\eta} \downarrow &\implies &\hat{L}_{\eta}^S \uparrow > \hat{L}_{\eta}^D \uparrow &\implies &\hat{Y}_{\eta} \uparrow &\implies &\hat{\lambda}_{\eta} \downarrow &\implies \\
			\hat{W}_{\eta} \downarrow &\implies & \Delta \hat{Q}_{\eta} ; \Delta \hat{P}_{\eta} \to 0 % \uparrow \downarrow 
		\end{matrix} \label{eq:flow}
	\end{align}
	
	\begin{comment}

	\begin{alignat}{4}
	\hat{Z}_{M} \uparrow      &\implies \hat{R}_{}\uparrow                                    &\implies \hat{I}_{\eta}^D \uparrow &\implies \hat{C}_{\eta} \downarrow &\implies \nonumber \\
	\hat{K}_{\eta} \downarrow &\implies \hat{L}_{\eta}^S \uparrow > \hat{L}_{\eta}^D \uparrow &\implies \hat{W}_{\eta} \downarrow &\implies & \nonumber \\
	\hat{Y}_{\eta} \uparrow   &\implies \hat{\lambda}_{\eta} \downarrow                       &\implies \hat{Q}_{\eta} \uparrow \downarrow ; \hat{P}_{\eta} \uparrow \downarrow & \text{} & \label{flow2}
\end{alignat}


	\begin{matrix}
	\hat{Z}_{M}      \uparrow   \implies &
	\hat{R}_{}       \uparrow   \implies &
	\hat{I}_{\eta}^D \uparrow   \implies &
	\hat{C}_{\eta}   \downarrow \implies \\
	\hat{K}_{\eta}   \downarrow \implies &
	\hat{L}_{\eta}^S \uparrow > \hat{L}_{\eta}^D \uparrow   \implies &
	\hat{W}_{\eta}   \downarrow \implies &
	\hat{Y}_{\eta}   \uparrow   \implies \\
	\hat{\lambda}_{\eta} \downarrow \implies &
	\hat{Q}_{\eta} \uparrow \downarrow \implies &
	\hat{P}_{\eta} \uparrow \downarrow & \text{}
\end{matrix} \label{flow}

	\end{comment}
	

% Together with the macroeconomic discussion, the regional variables indicate that the monetary policy shock will generate different reactions on structural different regions, as the initial hypothesis suggested. Notice on figures \ref{fig:ZMt-Y1t} and {fig:ZMt-Y2t} that the growth of production on region 1 is higher than that of region 2. As region 1 is more capital intensive, it will have advantage on the raise of the interest rate, since more capital will be demanded in this region and therefore household will be more inclined to investments in this region. On the other hand, region 2 is more labor intensive and therefore will be less impacted by interest rates, having a lower reaction on its production.

%On the regional point of view, variables indicate that the monetary policy shock will elicit diverse reactions across structurally distinct regions, aligning with the initial hypothesis. Observing Figures \ref{fig:ZMt-Y1t} and \ref{fig:ZMt-Y2t}, it is evident that the growth in production in region 1 surpasses that in region 2. This difference can be attributed to the fact that region 1 is more capital-intensive: as households decide for investment in the present and consumption in the future, the demand for production and subsequent supply of capital will benefit firms on region 1 more than in region 2, tha is more labor-intensive, leading to a comparatively lower impact from changes in interest rates, resulting in a less pronounced reaction in its production.

From a regional perspective, Figures \eqref{fig:paired_pos_irf} and \eqref{fig:paired_pos_irf2} show the variables of each region in the same plot, indicating that the monetary policy shock will elicit diverse reactions across structurally distinct regions, aligning with the initial hypothesis. The production growth in region 1 surpasses that in region 2. This difference can be attributed to the fact that region 1 is more capital-intensive. As households opt for investment in the present and consumption in the future, the demand for production and subsequent supply of capital benefit firms in region 1 more than in region 2, which is more labor-intensive. This leads to a comparatively lower impact from changes in interest rates in region 2, resulting in a less pronounced reaction in its production. \textcolor{blue}{}While it may initially seem counter-intuitive that higher nominal interest rates result in increased capital rental costs, a crucial aspect of this model should be noted. In this framework, the savings channel accessible to households is represented by investments, subsequently transformed into capital. The augmentation of investment and subsequent capital supply, in turn, contributes to an elevation in production.

Similarly, Figures \eqref{fig:eM-neg-irf}, \eqref{fig:eM-neg-irf2}, \eqref{fig:paired-neg-irf} and \eqref{fig:paired-neg-irf2} show a negative monetary policy shock, also set at 1\%. Note that all the reactions described earlier occur in the opposite direction in this scenario. Therefore, the negative monetary shock yields  a mirrored result if compared to the positive one, demonstrating the effectiveness of the interest rate as an instrument of monetary policy in both directions. %This result demonstrates that, in the present model, the monetary authority is capable of dealing with both problems of high inflation or low economic growth.

The results presented in this study are in line with the existing literature on regional economic dynamics. \textcite{bertanha_efeitos_2008}, although using a different method, explored the effects of monetary policy on regional economies and found that regions with diverse economic structures exhibit responses with different intensities to policy shocks. Similarly, \textcite{osterno_uma_2022}, using the DSGE methodology, also emphasize the importance of considering regional variables in monetary policy formulation. Overall, the convergence of our findings with previous researchs underscores the robustness of our analysis and contributes to a deeper understanding of regional economic dynamics.

\begin{comment}
	
	\begin{align}
		&\Delta \hat{Z}_{M}      >0 \implies
		 \Delta \hat{R}_{}       >0 \implies
		 \Delta \hat{I}_{\eta}^D >0 \implies
		 \Delta \hat{C}_{\eta}   <0 \implies \nonumber \\
		&\Delta \hat{K}_{\eta}   <0 \implies 
		 \Delta \hat{L}_{\eta}^S > \Delta \hat{L}_{\eta}^D \implies
		 \Delta \hat{W}_{\eta}   <0 \implies
		 \Delta \hat{Y}_{\eta}   >0 \implies \nonumber \\
		&\Delta \hat{\lambda}_{\eta} <0 \implies
		 \Delta \hat{Q}_{\eta} >0 ; <0 \implies
		 \Delta \hat{P}_{\eta} >0 ; <0
	\end{align}
	
	 &  &  &  & \hat{C}_{\eta 1} & \hat{C}_{\eta 2} &  & \hat{Y}_{} & \hat{Z}_{A\eta} & 

&  &  & \hat{\pi}_{} &  &  & \hat{\pi}_{\eta}

	
\end{comment}





\end{document}