% --------------------------------------------------
% DOCUMENT CLASS
% --------------------------------------------------

\documentclass[../thesis.tex]{subfiles}

\begin{document}

\newpage

\section{Introduction}\label{sec:introduction}

%\subsection*{Context}

The importance of macroeconomic modeling as a tool for studying the connections between monetary economy and the outcomes of a country's aggregates is undeniable, as stated by \textcite{gali_monetary_2015}. Considering as well that Brazilian regions possess heterogeneous economic matrices and sectors that respond in different ways to monetary authority decisions, as indicated by \textcite{bertanha_efeitos_2008}, the need for a structural model capable of relating macroeconomic variables to regional variables becomes evident.

In this context, the present research proposes the development of a macroeconomic model with regional extensions, using the DSGE methodology\footnote{Dynamic and Stochastic General Equilibrium.}, which can demonstrate the existing relationships among the various considered variables and present impulse response functions that illustrate these relationships. With this model, we aim to investigate the existing relationship between the nominal interest rate of the Brazilian economy and the level of regional gross domestic product.

%\subsection*{Problem and Justification}

The main issue to be investigated is the impact of monetary authority decisions --- especially changes in the nominal interest rate --- on regional macroeconomic variables, particularly the Gross Domestic Product (GDP) of a given Brazilian region (in this context, a region can be anything from a Municipality, a State, an Economic Region or any other composition of the country).

Given that Brazilian regions have distinct economic matrices (agriculture, industry, extraction, etc.), and within each of these specializations, some sectors are more labor-intensive while others are more capital-intensive, it is plausible to assume that regional diversity allows each region to react differently to changes in the interest rate, as demonstrated by \textcite{haddad_matriz_2017} and \textcite{osterno_uma_2022}.

% Given the problem, we need to determine how the study will be conducted. As this is a topic that combines knowledge from Macroeconomics and Regional Economics, it will be necessary to address the main concepts from both areas to then determine a methodology capable of integrating this content.

Regional Economics investigations often employ tools borrowed from Macroeconomics, as highlighted by \textcite{rickman_modern_2010}. Examples include the Leontief input-output model, the Walrasian general equilibrium applied model, and the system of macroeconometric equations. These instances demonstrate how models from one field can be adapted and utilized by the other.

In line with this notion, the objective of this work is to utilize a DSGE model (a commonly used tool in Macroeconomics) to establish relationships between macroeconomic variables and regional ones. Subsequently, Brazilian economic data will be employed to calibrate the model, enabling the derivation of impulse response functions that closely resemble the dynamics of the Brazilian economy. DSGE models have already been employed to address regional questions in excellent works such as \textcite{tamegawa_two-region_2012}, \textcite{tamegawa_constructing_2013}, \textcite{mora_fdi_2019}, \textcite{costa_junior_dsge_2022}, and \textcite{osterno_uma_2022}, to cite a few.

%\subsection*{Contributions}

% Numerous studies address the effects of national aggregates on regional variables, as the ones cited above, and these will be appropriately presented in Section \eqref{sec:literature-review}. Among them, \textcite{osterno_uma_2022} investigates the relations between monetary policy and regional variables using a DSGE model, but the path used here and there are different: while \textcite{osterno_uma_2022} uses a top-down approach, opening up the SAMBA model from \textcite{castro_samba_2015} to add regional variables, here the procedure is bottom-up: a regional DSGE model is created from zero to demonstrate the relations between macroeconomic and regional variables. For this reason, it is reasonable to foresee that the results may be different in strength, as some variables and relations present there are not here, but somewhat similar in the direction of the reaction of the existing variables.

% However, in these studies, we have not found one that specifically investigates the relationship between the national nominal interest rate and regional GDP.

Numerous studies have addressed the effects of national aggregates on regional variables, as those cited above, and these will be appropriately presented in Section \eqref{sec:literature-review}. Among them, \textcite{osterno_uma_2022} investigates the relations between monetary policy and regional variables using a DSGE model. However, the approach taken here differs from theirs: while \textcite{osterno_uma_2022} employs a top-down approach, augmenting the SAMBA model from \textcite{castro_samba_2015} to include regional variables, our approach is bottom-up. We create a regional DSGE model from scratch to illustrate the relationships between macroeconomic and regional variables. Consequently, it is reasonable to expect that the results may differ in intensity, as some variables and relationships present in their study are not replicated here. Nevertheless, the direction of the reactions of the existing variables may exhibit some similarity.

The significance of this work can be identified by recognizing that, given the diversity of Brazilian regions, it is not plausible that a single macroeconomic variable will have the same effect in each of them (or at least not with the same intensity). Thus, a tool capable of quantifying the regional effect of a macroeconomic variable is an important addition to economic literature, as it investigates the transmission mechanisms of monetary policy to the regional aggregates. Additionally, it also adds to the array of policy evaluation instruments, such that various economic agents can use this tool to determine the conduct of their own internal policies. For example, banks can quantify the credit interest rate for a specific region based on the projected interest rate of the economy, considering the needs and potential development of each region separately from the rest of the country.

%\subsection*{Objectives}

The main objective is to create a DSGE model capable of relating a macroeconomic variable (the nominal interest rate) to a regional variable (the Gross Domestic Product of a Brazilian region), in order to assess the impact of an expansionary (or contractionary) monetary policy on a specific Brazilian region and the magnitude of that impact.

The specific objectives are
\begin{enumerate*}[label=(\arabic*)]
	\item elaborate a New Keynesian DSGE model with households, firms, monetary authority, price stickiness, productivity and monetary shocks and two regions (the main region and the rest of the country) to verify if the nominal interest rate determined by the monetary authority influences the regional GDP;
	% \item determine which variables must be regionalized in order to make a regional environment in order to verify if two regions may have different responses to the monetary policy shocks;
	\item produce the impulse response functions (IRF) and analyse the results of the regional model.
\end{enumerate*}

%\subsection*{Organization}

The other sections are organized as follows. Section \eqref{sec:literature-review} summarizes the related literature. Section \eqref{sec:methodology} describes the proposed regional DSGE model. Section \eqref{sec:results} presents the results and discussion. Finally, Section \eqref{sec:final-remarks} provides a summary of what was learned and outlines the next challenges. Additionally, I have included an appendix where some details and results are clarified.


\end{document}