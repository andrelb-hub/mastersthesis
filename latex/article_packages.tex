% --------------------------------------------------
% DOCUMENT MARGINS:
% --------------------------------------------------

\usepackage[
	a4paper, 
	left   = 2.5cm, 
	right  = 2.5cm,
	top    = 3cm, 
	bottom = 2.5cm
]{geometry}

% --------------------------------------------------
% PAGE LAYOUT:
% --------------------------------------------------

\usepackage{fancyhdr}
\pagestyle{fancy}
\fancyhf{} % clear all header and footer fields
\fancyhead[R]{\thepage}
\renewcommand{\headrulewidth}{0.0pt}
%\renewcommand{\footrulewidth}{0.0pt}
\setlength{\headheight}{15pt}

% --------------------------------------------------
% INPUT ENCODING:
% --------------------------------------------------

\usepackage[utf8]{inputenc}

% --------------------------------------------------
% FONT ENCODING:
% --------------------------------------------------

\usepackage[T1]{fontenc}

% --------------------------------------------------
% LANGUAGE:
% --------------------------------------------------

\usepackage[
    brazilian,
    english,
]{babel}

% --------------------------------------------------
% FONT:
% --------------------------------------------------

\usepackage{mathpazo}
% \usepackage{times}
% \usepackage{palatino} 
% \usepackage{fontspec}
% \setmainfont{Arial}

% --------------------------------------------------
% TIPEWRITTER FONT ENVIRONMENT:
% --------------------------------------------------

%\usepackage{fontspec}
%\newfontfamily\myfont[]{lmtt}
%\newenvironment{myfont}{\myfont}{\par}

% --------------------------------------------------
% SECTION FORMAT
% --------------------------------------------------

\usepackage{sectsty}
\sectionfont{\fontsize{12}{15}\selectfont\uppercase}
\subsectionfont{\normalfont\fontsize{12}{15}\selectfont\uppercase}
\subsubsectionfont{\normalfont\fontsize{12}{15}\selectfont}

% --------------------------------------------------
% SPACES BETWEEN LINES:
% --------------------------------------------------

\usepackage{setspace}
\onehalfspacing
%\singlespacing
%\doublespacing

% --------------------------------------------------
% PARAGRAPHS:
% --------------------------------------------------

\setlength{\parindent}{1cm}

% --------------------------------------------------
% FIRST PARAGRAPH ALSO 'INDENTED':
% --------------------------------------------------

\usepackage{indentfirst}

% --------------------------------------------------
% SPACES BETWEEN PARAGRAPHS:
% --------------------------------------------------

\setlength{\parskip}{10pt}

% --------------------------------------------------
% TEXT MARGINS: (cover page text)
% --------------------------------------------------

\usepackage{changepage}

% --------------------------------------------------
% JUSTIFIED TEXT: 
% --------------------------------------------------

\usepackage{ragged2e}

% --------------------------------------------------
% SPACE BETWEEN MAIN TEXT AND FOOTNOTES:
% --------------------------------------------------

\setlength{\skip\footins}{1cm}

% --------------------------------------------------
% SPACE BETWEEN FOOTNOTES:
% --------------------------------------------------

\setlength{\footnotesep}{1.5pc}

% --------------------------------------------------
% SPACE BETWEEN FOOTNOTE NUMBER AND TEXT:
% --------------------------------------------------

\usepackage[hang]{footmisc}
\setlength{\footnotemargin}{4mm}

% --------------------------------------------------
% FOOTNOTE MISC
% --------------------------------------------------

\usepackage{footmisc} % Load the footmisc package

% --------------------------------------------------
% WATERMARK (MARCA D'ÁGUA)
% --------------------------------------------------

\usepackage[firstpage = true]{background}

\backgroundsetup{
scale    = 1,
opacity  = 1,
angle    = 0,
contents = {\includegraphics[width=\paperwidth]{marca_dagua_ufpr}}
}



% --------------------------------------------------
% TABLE OF CONTENTS DEPTH:
% --------------------------------------------------

\setcounter{tocdepth}{2}

% --------------------------------------------------
% TABLE OF CONTENTS FORMAT
% --------------------------------------------------

% Set font formatting for table of contents
\usepackage{tocloft}

% Adjust the width of the section number box in TOC
% \setlength{\cftsecnumwidth}{3cm} % Adjust as needed
% \setlength{\cftsubsecnumwidth}{3cm} % Adjust as needed

% centered ToC title:
\renewcommand{\cfttoctitlefont}{\hspace*{\fill}\large\scshape\bfseries}
\renewcommand{\cftaftertoctitle}{\hspace*{\fill}}

% centered LoT title:
\renewcommand{\cftlottitlefont}{\hspace*{\fill}\large\scshape\bfseries}
\renewcommand{\cftafterlottitle}{\hspace*{\fill}}

% centered LoF title:
\renewcommand{\cftloftitlefont}{\hspace*{\fill}\large\scshape\bfseries}
\renewcommand{\cftafterloftitle}{\hspace*{\fill}}

% ToC items:
%\renewcommand{\cftsecfont}{\bfseries\uppercase}
%\renewcommand{\cftsubsecfont}{\normalfont\uppercase}

% --------------------------------------------------
% ITEMS of TABLE OF CONTENTS FORMAT
% --------------------------------------------------

\usepackage{etoolbox}

 \makeatletter
% \patchcmd{<cmd>}{<search>}{<replace>}{<success>}{<failure>}
% \patchcmd{\tableofcontents}{\contentsname}{\MakeUppercase{\contentsname}}{}{}
\patchcmd{\l@section}{#1}{\MakeUppercase{#1}}{}{}% Sections use UPPERCASE in ToC
\patchcmd{\l@subsection}{#1}{\MakeUppercase{#1}}{}{}% Subsections use UPPERCASE in ToC
\makeatother

% --------------------------------------------------
% SUBTABLE OF CONTENTS INSIDE SECTIONS:
% --------------------------------------------------

\usepackage{etoc}

% --------------------------------------------------
% ENUMERATE INLINE:
% --------------------------------------------------

\usepackage[inline]{enumitem}

% USAGE:
%\begin{enumerate*}[label=(\arabic*)]
%	\item
%	...
%\end{enumerate*}

% --------------------------------------------------
% COLOURS:
% --------------------------------------------------

\usepackage[dvipsnames]{xcolor}

% --------------------------------------------------
% MULTIFIGURES AND SUBFIGURES
% --------------------------------------------------

\usepackage{caption} % for command \caption*{}

% Set global caption justification to left and font size to small
\captionsetup{font=small,
              % justification=raggedright, 
              % singlelinecheck=false
	      }

\usepackage{subcaption}

% --------------------------------------------------
% QUOTATIONS
% --------------------------------------------------

\usepackage{csquotes}

% --------------------------------------------------
% TABLES
% --------------------------------------------------

% new table package:
\usepackage{tabularray}

% commands: \toprule, \midrule, \bottomrule
\usepackage{booktabs}

% line break in table cell:
\usepackage{makecell}
% multirow:
\usepackage{multirow}

% long tables:
\usepackage{longtable}

% --------------------------------------------------
% LIPSUM TEXT
% --------------------------------------------------

\usepackage{lipsum}
\setlipsum{
	par-before = \begingroup\color{gray},
	par-after  = \endgroup
}

% --------------------------------------------------
% MATH ENVIRONMENT
% --------------------------------------------------

\usepackage[
	fleqn % align all equations left
]{amsmath}

% \setlength{\mathindent}{0cm}

\usepackage{empheq}

% always after \usepackage{amsmath}:
\usepackage{mathtools}

% --------------------------------------------------
% MATH SYMBOLS
% --------------------------------------------------

\usepackage{amssymb}

% --------------------------------------------------
% NUMBER THE EQUATIONS WITHIN THE SECTION
% --------------------------------------------------

\numberwithin{equation}{section}

% --------------------------------------------------
% SPACE BETWEEN EQUATIONS
% --------------------------------------------------

\setlength{\jot}{5pt}

% --------------------------------------------------
% MATH FONTS
% --------------------------------------------------

% https://ctan.org/pkg/mathalpha?lang=en

\usepackage[
	frak = euler, %boondox, %(for fraktur fonts)
	scr  = cm, %euler, %stixplain, %pxtx, %cm, %zapfc, %(for calligraphic fonts)
	cal  = boondoxo, %boondox, %(for script fonts)
	bb   = bboldx, %pazo, %boondox, %dsfontsans, % (for blackboard bold (double-struck) fonts)
	%bfscr, % force \mathscr to point to the bold version.
	%bfcal, % force \mathcal to point to the bold version.
	%bfbb,  % force \mathbb  to point to the bold version.
]{mathalpha}

% --------------------------------------------------
% BOLD MATH SYMBOLS
% --------------------------------------------------

\usepackage{bm}
% command: \bm{}

% --------------------------------------------------
% VERBATIM ENVIRONMENT
% --------------------------------------------------

\usepackage{verbatim}
% usage: \begin{verbatim} code \end{verbatim}

% verbatim with line breaks:
\usepackage{spverbatim}
% usage: \begin{spverbatim} code \end{spverbatim}

% --------------------------------------------------
% CANCEL LINE IN EQUATIONS
% --------------------------------------------------

\usepackage{cancel}
% https://ctan.org/pkg/cancel?lang=en

% --------------------------------------------------
% NICE FRACTIONS INLINE
% --------------------------------------------------

\usepackage{xfrac}
% usage: \sfrac[⟨instance⟩]{⟨num⟩}[⟨sep⟩]{⟨denom⟩}

% --------------------------------------------------
% THEOREMS
% --------------------------------------------------

\usepackage{amsthm}

% DEFINITION:
\theoremstyle{definition}
\newtheorem{definition}{Definition}[section]

% THEOREM:
\theoremstyle{plain}
\newtheorem{theorem}{Theorem}[section]

% LEMMA:
\theoremstyle{plain}
\newtheorem{lemma}{Lemma}[section]

% COROLLARY:
\theoremstyle{plain}
\newtheorem{corollary}{Corollary}[lemma]

% \blacksquare to end the proof:
\renewcommand\qedsymbol{$\blacksquare$}

% --------------------------------------------------
% MATH COMMANDS
% --------------------------------------------------

% differential d:
\DeclareMathOperator{\dif}{d}

% subject to:
\DeclareMathOperator{\st}{s.t.}

% expectation symbol:
\newcommand{\E}[1][t]{{\mathbb{E}_{#1}}}

% subscript text: $_t$
\newcommand{\subt}[1][t]{{_{#1}}}

% superscript text: $^$
\newcommand{\supt}[1]{{^{#1}}}

% the greek letter omicron:
\newcommand{\omicron}{o}

% --------------------------------------------------
% OTHER COMMANDS
% --------------------------------------------------

% the name of the command:
\newcommand{\com}[1]{\texttt{\textbackslash #1}}

% the name of the software:
% \newcommand{\dynare}{\texttt{Dynare}}
\newcommand{\dynare}{
	%\colorbox{lightgray}
	{$\mathsf{Dynare}$}}
\newcommand{\matlab}{\texttt{MATLAB}}
\newcommand{\octave}{\texttt{OCTAVE}}

% checkmark and xmark:
% usage: \cmark, \xmark 
\usepackage{pifont}% http://ctan.org/pkg/pifont
\newcommand{\cmark}{\textcolor{purple}{\ding{51}}}
\newcommand{\xmark}{\textcolor{gray}{\ding{55}}}

% --------------------------------------------------
% INDEX
% --------------------------------------------------

% USAGE: \index{exemplo}
% options: configure: build: Compile & View | txs:///pdflatex | txs:///makeindex
\usepackage{imakeidx}
\makeindex[intoc] % in table of contents
\indexsetup{othercode=\renewcommand{\indexspace}{\string\textbar\ }}

% --------------------------------------------------
% HYPERTEXT LINKS
% --------------------------------------------------

\usepackage[
	colorlinks = true, 
	linkcolor  = teal, 
	filecolor  = teal, 
	urlcolor   = teal, 
	citecolor  = teal
]{hyperref}

% --------------------------------------------------
% BOOKMARKS
% --------------------------------------------------

\usepackage[open,openlevel=1]{bookmark}

% --------------------------------------------------
% ACRONYMS
% --------------------------------------------------

\usepackage{acronym} 
% usage: \ac{TLA}

% --------------------------------------------------
% COLOR BOXES
% --------------------------------------------------

\usepackage[most]{tcolorbox}
% usage: \begin{tcolorbox}[colback=red!5!white,colframe=red!75!black] My box.\end{tcolorbox}
% usage: \begin{tcolorbox}[colback=blue!5!white,colframe=blue!75!black,title=My title] My box with my title. \end{tcolorbox}

% --------------------------------------------------
% TODO NOTES
% --------------------------------------------------

\setlength{\marginparwidth}{2cm}

\usepackage[
	%disable, % OPTION TO DISABLE THE TODO NOTES.
	textsize = footnotesize,
	color = orange!10,
]{todonotes}

% todo command:
% \todo[inline]{sua tarefa aqui.}

% --------------------------------------------------
% COMMENT ENVIRONMENT
% --------------------------------------------------

\usepackage{comment}