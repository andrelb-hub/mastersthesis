\documentclass[12pt]{article}

% margens do documento:
\usepackage[a4paper, left=3cm, right=3cm, top=3cm, bottom=3cm]{geometry}

% input encoding:
\usepackage[utf8]{inputenc}

% font encoding
\usepackage[T1]{fontenc}

% language:
\usepackage[english, brazilian]{babel} %
\hyphenation{pa-la-vras pa-ra hi-fe-ni-zar}

%%%%%%%%%%%%%%%%%%%%%%%%%%%%%%%%%%%%%%%%%%%%%%%%%%%%%%%%%%%%
%%%%%%%%%%%%%%%%%%%%%%%%% COLOURS %%%%%%%%%%%%%%%%%%%%%%%%%%
%%%%%%%%%%%%%%%%%%%%%%%%%%%%%%%%%%%%%%%%%%%%%%%%%%%%%%%%%%%%

% colours:
\usepackage[dvipsnames]{xcolor}

% espaço entre linhas:
\usepackage{setspace}
%\singlespacing
%\onehalfspacing
\doublespacing

% parágrafo:
\setlength{\parindent}{1cm}

% espaço entre parágrafos:
\setlength{\parskip}{1pt}

% primeiro parágrafo também "indented":
\usepackage{indentfirst}

% fonte:
\usepackage{palatino} 

% space between text and footnotes:
\setlength{\skip\footins}{1cm}

% space between footnotes:
\setlength{\footnotesep}{1.5pc}

% space between footnote number and text:
\usepackage[hang]{footmisc}
\setlength{\footnotemargin}{2mm}

% references: biber
\usepackage[style        = abnt, 
			sorting      = nyt, 
			backref      = true, 
			backend      = biber, 
			citecounter  = true, 
			backrefstyle = three, 
			url          = true, 
			maxbibnames  = 99, 
			mincitenames = 1, 
			maxcitenames = 2, 
			hyperref     = true, 
			giveninits   = true, 
			uniquename   = false, 
			uniquelist   = false]{biblatex}

\addbibresource{../ref.bib}

% \graphicspath{ {../images/} }


% advanced facilities for inline and display quotations:
\usepackage{csquotes}

%%%%%%%%%%%%%%%%%%%%%%%%%%%%%%%%%%%%%%%%%%%%%%%%%%
%%%%%%%%%%%%%%%       TABLES       %%%%%%%%%%%%%%%
%%%%%%%%%%%%%%%%%%%%%%%%%%%%%%%%%%%%%%%%%%%%%%%%%%

\usepackage{tabularray}

% Tabelas:
% line break in table cell:
\usepackage{makecell}
% multirow:
\usepackage{multirow}

% long tables:
\usepackage{longtable}

%%%%%%%%%%%%%%%%%%%%%%%%%%%%%%%%%%%%%%%%%%%%%%%%%%
%%%%%%%%%%%%%%%    RANDOM TEXT     %%%%%%%%%%%%%%%
%%%%%%%%%%%%%%%%%%%%%%%%%%%%%%%%%%%%%%%%%%%%%%%%%%

% random text:
\usepackage{lipsum}
\setlipsum{
	par-before = \begingroup\color{gray},
	par-after  = \endgroup
	}

%%%%%%%%%%%%%%%%%%%%%%%%%%%%%%%%%%%%%%%%%%%%%%%%%%
%%%%%%%%%%%%%%%        MATHS       %%%%%%%%%%%%%%%
%%%%%%%%%%%%%%%%%%%%%%%%%%%%%%%%%%%%%%%%%%%%%%%%%%

% math:
\usepackage{amsmath}

%%%%%%%%%%%%%%%%%%%%%%%%%%%%%%%%%%%%%%%%%%%%%%%%%%
%%%%%%%%%%%%%%%      COMMANDS      %%%%%%%%%%%%%%%
%%%%%%%%%%%%%%%%%%%%%%%%%%%%%%%%%%%%%%%%%%%%%%%%%%

% differential d:
\newcommand*\diff{\mathop{}\!\mathrm{d}}

% $t$ subscript:
\newcommand{\T}[2][t]{#2_{#1}}

% $t+1$ subscript:
\newcommand{\TT}[2][t]{#2_{#1+1}}

% $ss$ subscript:
\newcommand{\st}[2][ss]{#2_{#1}}

% Dynare:
\newcommand{\dynare}{\texttt{Dynare}}
\newcommand{\matlab}{\texttt{MATLAB}}
\newcommand{\octave}{\texttt{OCTAVE}}

% number the equations within the section:
\numberwithin{equation}{section}

% number-set fonts:
\usepackage{amsfonts}

% math symbols
\usepackage{amssymb}

% Lagrangian L: \mathscr{L}
\usepackage{mathrsfs}

% index: \index{exemplo}
% options: configure: build: Compile & View | txs:///pdflatex | txs:///makeindex
\usepackage{imakeidx}
\makeindex[intoc] % in table of contents
\indexsetup{othercode=\renewcommand{\indexspace}{\string\textbar\ }}

% hyperlinks and setup:
\usepackage[colorlinks = true, 
			linkcolor  = teal, 
			filecolor  = teal, 
			urlcolor   = teal, 
			citecolor  = teal]{hyperref}

%%%%%%%%%%%%%%%%%%%%%%%%%%%%%%%%%%%%%%%%%%%%%%%%%%%%%%%%%%%%
%%%%%%%%%%%%%%%%%%%%%%%%% THEOREMS %%%%%%%%%%%%%%%%%%%%%%%%%
%%%%%%%%%%%%%%%%%%%%%%%%%%%%%%%%%%%%%%%%%%%%%%%%%%%%%%%%%%%%

\usepackage{amsthm}
\theoremstyle{definition}
\newtheorem{definition}{}[section]

%%%%%%%%%%%%%%%%%%%%%%%%%%%%%%%%%%%%%%%%%%%%%%%%%%%%%%%%%%%%%%%
%%%%%%%%%%%%%%%%%%%%%%%%%%%%%%%%%%%%%%%%%%%%%%%%%%%%%%%%%%%%%%%

\title{\LARGE Título Provisório: \\ 
	   [1ex] \large
	   Política Monetária e Economia Regional: \\
	   Um Modelo Estrutural para Análise dos Impactos \\
	   da Taxa de Juros sobre uma Região Brasileira}

\author{ \\ \\ \LARGE André Luiz Brito
	\footnote{
		\href{mailto:andreluizmtg@gmail.com}
		{andreluizmtg@gmail.com}}}

\date{}

%%%%%%%%%%%%%%%%%%%%%%%%%%%%%%%%%%%%%%%%%%%%%%%%%%%%%%%%%%%%%%%
%%%%%%%%%%%%%%%%%%%%%%%%%%%%%%%%%%%%%%%%%%%%%%%%%%%%%%%%%%%%%%%

\begin{document}

{\doublespacing
 \maketitle
 }

\vfill

\begin{center}
	Curitiba, 01 de março de 2023
\end{center}


\newpage

{\singlespacing
	\begin{abstract}

%\textit{(resumo do projeto de dissertação.)}

O presente projeto de pesquisa propõe criar um modelo estrutural utilizando a metodologia DSGE (\textit{Dynamic and Stochastic General Equilibrium} ou Equilíbrio Geral Dinâmico e Estocástico) para investigar se existe correlação entre uma variável macroeconômica (a taxa de juros nominal) e uma variável regional (o nível de produção de uma região brasileira). Caso exista, a pesquisa também pretende dimensionar esta correlação.
		
	\end{abstract}
}

\newpage

{\selectlanguage{english}
	\singlespacing
	\begin{abstract}

This research project aims to create a structural model using the DSGE (Dynamic and Stochastic General Equilibrium) methodology to investigate if there is a correlation between a macroeconomic variable (the nominal interest rate) and a regional variable (the level of production of a Brazilian region). If such a correlation exists, the research also intends to quantify it.

	\end{abstract}
}

\newpage

{\singlespacing
	\tableofcontents
}

\newpage

\section{Introdução}\label{sec:introdução}

%\textit{(breve contextualização do tema.)}

A importância da modelagem macroeconômica como ferramenta para estudar as ligações entre a economia monetária e os resultados dos agregados de um país é indiscutível, como afirma \textcite{gali_monetary_2015}. Considerando também que as regiões brasileiras possuem matrizes e setores econômicos heterogêneos que respondem de diferentes formas às decisões da autoridade monetária, como apontam \textcite{bertanha_efeitos_2008}, torna-se evidente a necessidade de um modelo estrutural capaz de relacionar as variáveis macroeconômicas às variáveis regionais.

Neste sentido, o presente projeto de pesquisa propõe o desenvolvimento de um modelo macroeconômico com desdobramentos regionais, utilizando a metodologia DSGE\footnote{$\,$ \textit{Dynamic and Stochastic General Equilibrium} ou Equilíbrio Geral Dinâmico e Estocástico.}, que é capaz de demonstrar as relações existentes entre as diversas variáveis consideradas e apontar os mecanismos de transmissão da política monetária para a economia regional que será objeto de estudo.

Com este modelo, pretendemos confirmar se existe correlação entre a taxa de juros nominal da economia e o nível de produção regional. Caso exista, pretendemos determinar a dimensão desta correlação, para então concluirmos se é uma correlação forte ou fraca.

\newpage

\section{Problema e Justificativa} \label{sec:problemaEjustificativa}

%\textit{(apresentar o problema principal de investigação e descrever os principais argumentos que justificam o desenvolvimento do tema de dissertação proposta; explicar a contribuição do estudo para a literatura da área)}

O problema principal a ser investigado é se existe correlação e, caso exista, qual a dimensão da correlação entre as decisões da autoridade monetária, em especial as alterações da taxa de juros nominal, e as variáveis macroeconômicas regionais, em especial o produto interno bruto (PIB) de uma dada região brasileira (um estado, por exemplo).

\sloppy Dado que as regiões brasileiras possuem matrizes econômicas distintas (agricultura, indústria, extrativismo, etc.) e, dentro de cada uma destas especializações, alguns setores são mais intensivos em mão-de-obra e outros em capital, é plausível supor que a diversidade regional permite a cada região reagir de forma distinta às alterações da taxa de juros.

Dado o problema, precisamos determinar como o estudo será conduzido. Por se tratar de um tema que une conhecimentos de Macroeconomia e Economia Regional, será necessário que sejam abordados os principais conceitos das duas áreas para então determinar uma metodologia capaz de unir todo este conteúdo.

As investigações da Economia Regional tomam emprestado o ferramental da Macroeconomia, como aponta \textcite{rickman_modern_2010}: o modelo insumo-produto de Leontief, o modelo Walrasiano de equilíbrio geral aplicado e o sistema de equações macroeconométricas são alguns exemplos de como modelos de uma área de estudo foram adaptados e passaram a ser usada pela outra. Neste sentido, a proposta para o presente trabalho é utilizar um modelo estrutural (que é uma ferramenta de macroeconomia) para determinar as relações entre uma variável macro e as variáveis regionais para, em um segundo momento, utilizar os dados da economia brasileira para determinar o nível de correlação entre elas.

Diversos são os estudos que abordam os efeitos de agregados nacionais sobre variáveis regionais, e que serão devidamente apresentados na seção \ref{sec:revisaoDaLiteratura}. Nestes estudos, entretanto, não encontramos um que investigue de forma específica a quantificação da correlação entre a taxa de juros e o PIB regional.

A relevância do presente trabalho pode ser identificada ao lembrarmos que, dada a diversidade das regiões brasileiras, não é verossímil que uma única variável macroeconômica irá gerar o mesmo efeito em cada uma delas (ou pelo menos com a mesma intensidade). Assim, uma ferramenta capaz de quantificar o efeito regional de uma variável macroeconômica é uma importante adição à literatura econômica, ao demonstrar os mecanismos de transmissão do agregado nacional para o agregado regional. Além disso, é também uma adição ao rol de instrumentos de avaliação de política monetária, de tal forma que diversos agentes econômicos poderão utilizar esta ferramenta para determinar a condução de políticas internas. Por exemplo, bancos poderão quantificar a taxa de juros de crédito para uma determinada região em função da projeção da taxa de juros, levando em consideração a necessidade e potencial de desenvolvimento de cada região e a política monetária.

\newpage

\section{Objetivos}\label{sec:objetivos}

\subsection{Objetivo Geral}\label{subsec:objetivoGeral}

O objetivo do presente trabalho é criar um modelo estrutural capaz de relacionar uma variável macroeconômica (a taxa de juros nominal) a uma variável regional (o produto de uma região brasileira), com a finalidade de investigar se existe e, caso exista, qual a dimensão da influência desta variável macroeconômica sobre a variável regional.

\subsection{Objetivos Específicos}\label{subsec:objetivosEsp}

Os objetivos específicos são:

\begin{itemize}
	\item Demonstrar se existe ou não correlação entre taxa de juros nominal e o nível de produção regional.
	\item Caso exista, quantificar a correlação entre taxa de juros nominal e o nível de produção regional.
	\item Elaborar um modelo capaz de demonstrar os mecanismos de transmissão da política monetária para as variáveis regionais.
	\item Avaliar a capacidade do modelo de descrever as relações investigadas.
\end{itemize}

\newpage

\section{Revisão da Literatura}\label{sec:revisaoDaLiteratura}

%\textit{(texto discursivo sobre trabalhos da literatura que são considerados relevantes – estado da arte - para a dissertação. Deve servir para caracterizar o embasamento do leitor no tema de sua dissertação. Para Mestrado, cerca de 7 páginas, abrangendo publicações em periódicos indexados ou livros pelo menos nos últimos 10 anos)}

\subsection{Macroeconomia e Economia Regional}

A avaliação de \textcite{rickman_modern_2010} sobre a importância da ligação entre Macroeconomia e Economia Regional foi em um época que ainda não era comum a utilização de modelos estruturais para investigar questões regionais. De lá para cá, vários estudos contemplam esta ligação.

Inicialmente, apresentamos os dois trabalhos que foram a inspiração para o presente tema. O primeiro trata da avaliação dos impactos da política fiscal sobre o estado de Goiás, levando em consideração os demais estados da nação, elaborado por \textcite{costa_junior_dsge_2022}. Nele, os autores desenvolvem uma estrutura regionalizada e aberta, individualizando um estado brasileiro dos demais, considerando uma autoridade fiscal nacional e outra estadual; as despesas e receitas estaduais são desagregadas e, com isso, os autores buscam identificar se existem diferenças entre os impactos de uma desoneração fiscal no estado objeto do estudo frente aos demais. Com o modelo calibrado para os dados de 2003 a 2019, os autores demonstram haver sim diferença na performance estadual em função da distinção da desoneração se dar no estado ou no resto do país.

O segundo também apresenta um modelo estrutural, mas com objetivo de avaliar se existe diferença nos efeitos do investimento estrangeiro direto (IED), levando em consideração onde ele é aplicado. O modelo estrutural elaborado por \textcite{mora_fdi_2019} contempla uma economia aberta com a região principal (Bogotá, 25\% do PIB nacional) e o resto do país (Colômbia), dois tipos de agentes\footnote{$\,$ agentes Ricardianos e não-Ricardianos.}, formação de hábitos, custos de ajuste de capital, além dos elementos típicos de um modelo Novo Keynesiano (NK)\footnote{$\,$ rigidez de preços nominais, competição monopolística, não neutralidade da política monetária no curto prazo.}. Com calibração do modelo para os dados de 2002 a 2015, os autores demonstram que existe sim diferença entre os efeitos do IED em função da região em que ele é aplicado, de tal forma que, quando é aplicado no resto do país, há efeitos de crescimento que se espalham por todo o país, por transbordamento, inclusive para região principal.

Os dois trabalhos buscam, apesar de tratarem de causas distintas (política fiscal e IED), verificar se existem diferenças quando a causa se dá em uma das duas diferentes regiões modeladas. Além disso, compartilham da mesma proposta de modelagem, a de um modelo estrutural de equilíbrio geral dinâmico e estocástico, genericamente denominados pela literatura como DSGE \textit{(Dynamic and Stochastic General Equilibrium)}. E era este avanço que \textcite{rickman_modern_2010} queria ver acontecer: a utilização de modelos estruturais para responder questões regionais.

\subsection{Modelagem Macroeconômica}

O referencial teórico da modelagem estrutural é extenso, uma vez que, a partir de um modelo de equilíbrio geral, é possível formular perguntas variadas, desde as citadas acima até questões sobre, por exemplo, mercado de trabalho, como feito por \textcite{ribeiro_alongamento_2023}, mercado imobiliário, por \textcite{albuquerquemello_mercado_2018} e também sobre desmatamento, por \textcite{pereira_desmatamento_2013}. Como comentou \textcite{solis-garcia_ucb_2022}: \textit{se você possui uma ideia econômica coesa, você pode colocar em termos de um modelo DSGE}\footnote{$\,$ \textit{"If you have a cohesive economic idea, you can put it in terms of a DSGE model."}}.

Em termos teóricos, deve fazer parte do referencial os trabalhos de \textcite{costa_junior_understanding_2016}, \textcite{bergholt_basic_2012} e \textcite{gali_monetary_2015}, os quais conduzem o leitor no desenvolvimento dos modelos estruturais. \textcite{costa_junior_understanding_2016} aborda o desenvolvimento utilizando a teoria dos Ciclos Reais de Negócios (RBC, \textit{Real Business Cycles theory}) para então adicionar os elementos da teoria Novo Keynesiana (NK, \textit{New Keynesian theory}). \textcite{bergholt_basic_2012} discute os principais elementos de um modelo novo Keynesiano e demonstra também a programação necessária para executar o modelo no programa \dynare{}\footnote{$\,$ mais detalhes sobre o programa \dynare{} na seção \ref{sec:dynare}. }. \textcite{gali_monetary_2015} mostra a evolução de um modelo RBC para um NK, adicionando um nível a mais de complexidade ao modelo a cada capítulo.

\subsection{Modelagem Macroeconômica com Regiões}

No presente trabalho, o foco é a utilização da modelagem estrutural com regiões para investigar a existência de relação entre uma variável macroeconômica e uma regional.

Do rol de trabalhos que usam a metodologia estrutural com regiões, podemos citar o estudo de \textcite{tamegawa_two-region_2012}, que avalia os efeitos da política fiscal em função de duas regiões, utilizando um modelo com dois tipos de agentes, bancos, um governo nacional e outro regional. Usando parâmetros da literatura para calibrar o modelo, os resultados mostram que existe sim diferença entre os efeitos da política fiscal, em função de qual região está implementando-a. Importante observar que a diferença entre um modelo estrutural macroeconômico e um regional é que no primeiro as variáveis agregadas são consideradas apenas no nível nacional, enquanto que no segundo são consideradas tanto variáveis nacionais quanto variáveis regionais e, a depender do tamanho da região, estas não são capazes de afetar aquelas, como explica \textcite{tamegawa_constructing_2013}.

No mesmo sentido de demonstrar relações regionais, \textcite{pytlarczyk_estimated_2005} pesquisa os aspectos da União Monetária Européia (UME), com foco na economia alemã, a partir de um modelo estrutural com duas regiões; \textcite{gali_optimal_2005} também avaliam o funcionamento da UME, mas com um modelo em que as regiões formam um contínuo unitário, de tal forma que uma região não é capaz de afetar toda a economia. \textcite{alpanda_international_2014} usam um modelo de duas regiões para avaliar os efeitos dos choques financeiros nos EUA sobre a economia na região do euro.

Uma estrutura para avaliar a evolução econômica de uma região do Japão é elaborada por \textcite{okano_development_2015}, com o intuito de identificar as causas da estagnação da região de Kansai.

Mais recentemente, o artigo de \textcite{croitorov_financial_2020} busca identificar os \textit{spillovers}\footnotemark{}  entre regiões, construindo um modelo com três regiões: a área do Euro, os EUA e o resto do mundo. No mesmo sentido de investigar os \textit{spillovers}, \textcite{corbo_maja_2020} apresentam um modelo regional contemplando a Suécia e o resto do mundo.\footnotetext{$\,$ transbordamentos: efeitos que são transmitidos de uma região para outra, devido a um motivo exógeno como, por exemplo, o simples fato de serem regiões vizinhas.}

\subsection{Política Monetária}

Os modelos estruturais são amplamente utilizados pela literatura de macroeconomia para investigar os efeitos da política monetária sobre os agregados macroeconômicos, como ensina \textcite{gali_monetary_2015}. Neste sentido, importante adicionar ao referencial teórico os textos que tratam do desenvolvimento dos modelos que incorporam os elementos que descrevem as regras utilizadas pela autoridade monetária para determinar a taxa de juros da economia.

\textcite{smets_estimated_2003} e \textcite{smets_shocks_2007} apresentam modelos estruturais que avaliam choques de diversos tipos na zona do Euro e nos Estados Unidos, respectivamente; \textcite{walque_financial_2010} avaliam o papel do setor bancário na recuperação da liquidez do mercado, considerando endógenas as possibilidades de \textit{default} dos agentes.

\textcite{holm_transmission_2021} estudam a transmissão da política monetária para o consumo das famílias, estimando a resposta do consumo, da renda e da poupança. É utilizado um modelo novo-keynesiano com agentes heterogêneos \footnote{$\,$ \textit{HANK: heterogeneous agent New Keynesian.}}. Os resultados mostram que uma política monetária restritiva faz as famílias com menos liquidez reduzir o consumo quando a renda disponível começa a cair, enquanto que as famílias com média liquidez poupam menos ou emprestam mais. Também fica constatada a diferença nas alterações de consumo entre poupadores e tomadores em face de uma alteração da política monetária.

\textcite{capeleti_countercyclical_2022} avaliam os efeitos de expansões de crédito pró-cíclicas e contracíclicas dos bancos públicos sobre o crescimento econômico. O modelo implementa um setor bancário com bancos públicos e privados competindo em um oligopólio de Cournot. Os resultados mostram que a oferta de crédito público tem um efeito mais forte quando a política é contracíclica.

\textcite{soltani_investigating_2021} investigam os choques financeiros e monetários em variáveis macroeconômicas, dando especial atenção ao papel dos bancos. Para esta análise, o modelo considera uma economia com setor bancário. Os resultados mostram que a atividade bancária pode influenciar os efeitos de políticas econômicas.

\textcite{vinhado_politica_2016} utilizam um modelo com fricções financeiras para verificar a transmissão de política monetária para o setor bancário e a atividade econômica. Os resultados demonstram que o setor bancário exerce importante papel na atividade econômica e impacta os resultados da política monetária ao ter que alterar o \textit{spread} bancário em função das alterações da taxa de juros ou do nível de depósito compulsório.

\newpage

\section{Metodologia}\label{sec:metodologia}

%(\textit{deve ser o mais detalhado possível, abrangendo dados e métodos teóricos e/ou empíricos quando for o caso.})

\subsection{Modelo Macroeconômico e Regional}

Neste trabalho, usaremos a metodologia de modelagem macroeconômica, mais especificamente, a metodologia DSGE. Primeiro, será elaborado um modelo de equilíbrio geral dinâmico e estocástico. Para o nosso tema, vamos elaborar uma estrutura que contemplará os seguintes elementos:

\begin{itemize}
	\item o modelo será populado por quatro agentes distintos: uma família representativa, uma firma representativa produtora de um bem final, um contínuo de firmas representativas produtoras de bens intermediários e uma autoridade monetária.
	\item a família representativa maximiza a utilidade em função do consumo e do trabalho, submetida a uma restrição orçamentária composta por salário, juros do aluguel do capital e lucro das firmas.
	\item as firmas intermediárias produzem, cada uma delas, um único bem intermediário, todos eles com um grau de substituição imperfeita, de tal forma que estas firmas operam em competição monopolística. As firmas de bens intermediários irão maximizar o lucro a cada período, em função do nível de produção e da mão-de-obra disponível.
	\item a firma representativa produtora do bem final consumido pelas famílias, que é um agregado dos bens intermediários produzidos pelas firmas intermediárias, opera em competição perfeita e irá maximizar o lucro em função do nível de produção, o qual depende também da quantidade de mão-de-obra ofertada pelas famílias a cada período.
	\item a cada período, uma parcela das firmas de bens intermediários tem a oportunidade de atualizar os preços, e as demais perdem esta oportunidade, seguindo uma regra de \textcite{calvo_staggered_1983}. Este mecanismo irá gerar fricções nominais nos preços dos bens, alterando as relações de equilíbrio do sistema. São estas fricções que geram a não neutralidade da moeda no curto prazo, como explica \textcite[p.191]{costa_junior_understanding_2016}.
	\item haverá uma autoridade monetária, que irá alterar a taxa de juros da economia em função das flutuações da inflação e da produção do período anterior, no intuito de controlar o nível de preços e o crescimento, seguindo uma regra de \textcite{taylor_discretion_1993}.
	\item os choques estocásticos estarão presentes na produtividade das firmas e nas preferências da família representativa.
	\item para a regionalização do modelo, iremos utilizar um índice para a região a ser estudada e o restante do Brasil, de tal forma que teremos as famílias, a firma de bens finais e as firmas de bens intermediários de cada região.
	\item as famílias não terão mobilidade entre as regiões, mas os bens intermediários e finais terão, e esse será o elo para conectar as duas regiões.
\end{itemize}

Segundo, faremos o fechamento dos mercados, igualando demanda e oferta da mão-de-obra e dos bens. 

Terceiro, estipulando que o sistema tende para o equilíbrio no longo prazo, então haverá um estado estacionário no qual as variáveis deixarão de sofrer alterações, de tal forma que, para um dado $t \longrightarrow \infty$, teremos $\boldsymbol{X}_t = \boldsymbol{X}_{t+1} = \boldsymbol{X}_{ss} \implies \boldsymbol{\dot{X}} = 0$\footnote{$\,$ lembre que: $\boldsymbol{\dot{X}} = \frac{\partial \boldsymbol{X}}{\partial t}$}. Onde $\boldsymbol{X}$ é o vetor das variáveis do sistema e $ss$ é o índice indicando o estado estacionário.

Quarto, aplicaremos o método de log-linearização de \textcite{uhlig_toolkit_1999} para transformar o sistema de equações em um sistema linear.

Quinto, criaremos o arquivo contendo o sistema de equações lineares para que o programa \dynare{} calcule a solução do sistema e produza os gráficos de impulso-resposta em função dos choques estocásticos.

Sexto, faremos a coleta dos dados macroeconômicos e regionais do período de 1995 (início do plano) a 2020, usando os dados dos vinte primeiros anos para calcular os parâmetros do modelo para, então, verificar a aderência do modelo aos dados dos últimos seis anos.

\subsection{Dynare} \label{sec:dynare}

Como os modelos de equilíbrio geral dinâmico e estocástico podem conter dezenas ou até mesmo centenas de variáveis, é necessária a utilização de força bruta computacional para se obter um resultado. O programa \dynare{} foi criado por \textcite{adjemian_dynare_2022} para suprir esta necessidade.

O \dynare{} é um pacote que pode ser instalado no programa \matlab{} ou no \octave{}. A partir de um arquivo texto com as equações do sistema e as regras de choques estocásticos, o \dynare{} é capaz de calcular a solução do sistema e apresentar os gráficos de impulso-resposta em função dos choques.

\newpage

\section{Resultados Esperados}\label{sec:resultadosEsperados}

%\textit{(descrever os resultados e/ou produtos específicos que são esperados da dissertação.)}

A partir do presente projeto, pretendemos conduzir uma pesquisa que contemple a criação de um modelo estrutural para avaliação dos choques macroeconômicos sobre uma dada região brasileira e avaliar a dimensão da correlação entre a variável macro e a regional.

\newpage

\section{Cronograma}\label{sec:cronograma}

%\textit{(situar no tempo as etapas e atividades previstas no plano de trabalho até a data prevista para a defesa.)}

\noindent
\begin{tblr}{
		colspec={lccccccccc},
		hlines,
		vlines
	}
Etapa & mar & abr & mai & jun & jul & ago & set & out & nov   \\
pesquisa bibliográfica    & x & x &   &   &   &   &   &   &   \\
entrega do projeto        &   &   & x &   &   &   &   &   &   \\
modelagem                 &   &   & x &   &   &   &   &   &   \\
programação no \dynare    &   &   &   & x &   &   &   &   &   \\
análise dos gráficos      &   &   &   & x &   &   &   &   &   \\
seminário do projeto      &   &   &   & x &   &   &   &   &   \\
coleta dos dados          &   &   &   &   & x &   &   &   &   \\
tratamento dos dados      &   &   &   &   & x &   &   &   &   \\
cálculo dos parâmetros    &   &   &   &   & x &   &   &   &   \\
banca de qualificação     &   &   &   &   &   & x &   &   &   \\
projeção de cenários      &   &   &   &   &   & x &   &   &   \\
discussão dos resultados  &   &   &   &   &   &   & x &   &   \\
revisão do texto          &   &   &   &   &   &   & x & x &   \\
defesa da dissertação     &   &   &   &   &   &   &   &   & x \\
reuniões com o orientador & x & x & x & x & x & x & x & x & x \\
\end{tblr}

\newpage

\onehalfspacing

\printbibliography[heading=bibintoc]

\newpage

\appendix

\section*{Assinaturas}
\addcontentsline{toc}{section}{\protect\numberline{}Assinaturas}

\vspace{5cm}

{\singlespacing
	{\centering
		\noindent\rule{4in}{0.7pt}\\
		\noindent André Luiz Brito\\
		Mestrando\\
	}
}

\vspace{4cm}

Declaro que sou o orientador e aprovo este projeto de dissertação.

\vspace{4cm}

{\singlespacing
	{\centering
		\noindent\rule{4in}{0.7pt}\\
		Armando Vaz Sampaio\\
		Orientador\\
	}
}

\end{document}