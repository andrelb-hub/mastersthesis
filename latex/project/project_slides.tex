\documentclass[10pt]{beamer}

\usetheme[progressbar=frametitle]{metropolis}

\usepackage{appendixnumberbeamer}

% cor da barra de progresso:
\setbeamercolor{progress bar}{fg=orange,bg=white}

% largura da barra de progresso:
\makeatletter
\setlength{\metropolis@titleseparator@linewidth}{2pt}
\setlength{\metropolis@progressonsectionpage@linewidth}{2pt}
\setlength{\metropolis@progressinheadfoot@linewidth}{2pt}
\makeatother

\usepackage{booktabs}
\usepackage[scale=2]{ccicons}

\usepackage{pgfplots}
\usepgfplotslibrary{dateplot}

\usepackage{xspace}
\newcommand{\themename}{\textbf{\textsc{metropolis}}\xspace}

% itens transparentes:
\setbeamercovered{transparent}

% space between items:
\newlength{\wideitemsep}
\setlength{\wideitemsep}{\itemsep}\addtolength{\wideitemsep}{10pt}

% reconfigure itemize lists:
\let\olditem\item
\renewcommand{\item}{%
	\olditem\vspace{5pt}}

% tabela:
\usepackage{graphicx}

% line break in table cell:
\usepackage{makecell}

% references:
\usepackage[
style=abnt,
sorting=nyt,
backref=true,
backend=biber,
citecounter=true,
backrefstyle=three,
url=true,
maxbibnames=99,
mincitenames=1,
maxcitenames=2,
hyperref=true,
giveninits=true,
uniquename=false,
uniquelist=false
]{biblatex}

\addbibresource{../files/ref.bib}

\graphicspath{ {../images/} }

% define o idioma português:
\usepackage[brazil]{babel}

\usepackage{csquotes}

%%%%%%%%%%%%%%%%%%%%%%%%%%%%%%%%%%%%%%%%%%%%%%%%%%%%%%%%%%%%%%%
%%%%%%%%%%%%%%%%%%%%%%%%%%%%%%%%%%%%%%%%%%%%%%%%%%%%%%%%%%%%%%%

\title{Política Monetária e Economia Regional}
\subtitle{Um Modelo Estrutural para Análise dos Impactos 
	da Taxa de Juros sobre uma Região Brasileira}
\author{André Luiz Brito}
\institute{PPGDE-UFPR}
\date{13/06/2023}
\titlegraphic{\hfill\includegraphics[height=1cm]{marca_UFPR.png}}

%%%%%%%%%%%%%%%%%%%%%%%%%%%%%%%%%%%%%%%%%%%%%%%%%%%%%%%%%%%%%%%
%%%%%%%%%%%%%%%%%%%%%%%%%%%%%%%%%%%%%%%%%%%%%%%%%%%%%%%%%%%%%%%

\begin{document}
	
\begin{frame}
		
\maketitle
		
\end{frame}
	
%%%%%%%%%%%%%%%%%%%%%%%%%%%%%%%%%%%%%%%%%%%%%%%%%%%%%%%%%%%%%%%
%%%%%%%%%%%%%%%%%%%%%%%%%%%%%%%%%%%%%%%%%%%%%%%%%%%%%%%%%%%%%%%	
	
\begin{frame}{Conteúdo}
		
\setbeamertemplate{section in toc}[sections numbered]
\tableofcontents[hideallsubsections]
		
\end{frame}
	
%%%%%%%%%%%%%%%%%%%%%%%%%%%%%%%%%%%%%%%%%%%%%%%%%%%%%%%%%%%%%%%
%%%%%%%%%%%%%%%%%%%%%%%%%%%%%%%%%%%%%%%%%%%%%%%%%%%%%%%%%%%%%%%
	
\section{Introdução}
	
%%%%%%%%%%%%%%%%%%%%%%%%%%%%%%%%%%%%%%%%%%%%%%%%%%%%%%%%%%%%%%%
%%%%%%%%%%%%%%%%%%%%%%%%%%%%%%%%%%%%%%%%%%%%%%%%%%%%%%%%%%%%%%%
	
\begin{frame}{Introdução}
		
\begin{itemize}
	\item A modelagem macroeconômica é uma importante ferramenta para estudar as ligações entre a economia monetária e os resultados dos agregados de um país, \textcite{gali_monetary_2015}.
	\item As regiões brasileiras possuem matrizes e setores econômicos heterogêneos que respondem de diferentes formas às decisões da autoridade monetária, \textcite{bertanha_efeitos_2008}.
\end{itemize}

\end{frame}
	
%%%%%%%%%%%%%%%%%%%%%%%%%%%%%%%%%%%%%%%%%%%%%%%%%%%%%%%%%%%%%%%
%%%%%%%%%%%%%%%%%%%%%%%%%%%%%%%%%%%%%%%%%%%%%%%%%%%%%%%%%%%%%%%

\begin{frame}{Introdução}
	
	\begin{itemize}
		\item \textit{Na realidade, a maior parte das tolices já escritas e que se continuam a escrever sobre economia poderia ter sido poupada se todo economista fosse obrigado a expor suas ideias construindo um modelo matemático} --- \textcite[p.68]{simonsen_microeconomia_1979}.
	\end{itemize}
	
\end{frame}

%%%%%%%%%%%%%%%%%%%%%%%%%%%%%%%%%%%%%%%%%%%%%%%%%%%%%%%%%%%%%%%
%%%%%%%%%%%%%%%%%%%%%%%%%%%%%%%%%%%%%%%%%%%%%%%%%%%%%%%%%%%%%%%

\begin{frame}{Introdução}
	
	\begin{itemize}
		\item Proposta: desenvolver um modelo estrutural com desdobramentos regionais, utilizando a metodologia DSGE (\textit{Dynamic and Stochastic General Equilibrium}).
		\item Objetivo: verificar se existe correlação entre a taxa de juros nominal da economia (uma variável macroeconômica) e o nível de produção de uma região brasileira (uma variável regional). Existindo esta correlação, pretendemos quantificá-la.
	\end{itemize}
	
\end{frame}

%%%%%%%%%%%%%%%%%%%%%%%%%%%%%%%%%%%%%%%%%%%%%%%%%%%%%%%%%%%%%%%
%%%%%%%%%%%%%%%%%%%%%%%%%%%%%%%%%%%%%%%%%%%%%%%%%%%%%%%%%%%%%%%

\begin{frame}{O que é um modelo DSGE?}
	
	\begin{itemize}
		\item Os modelos DSGE são ferramentas utilizadas em macroeconomia para avaliar a relação existente entre as variáveis selecionadas pelo pesquisador.
		\item Tem como principais características um horizonte de tempo infinito e choques aleatórios sobre algumas variáveis de interesse.
	\end{itemize}
	
\end{frame}

%%%%%%%%%%%%%%%%%%%%%%%%%%%%%%%%%%%%%%%%%%%%%%%%%%%%%%%%%%%%%%%
%%%%%%%%%%%%%%%%%%%%%%%%%%%%%%%%%%%%%%%%%%%%%%%%%%%%%%%%%%%%%%%

\begin{frame}{Real Business Cycles Theory}
	
	\begin{itemize}
		\item Os modelos DSGE começaram a ser usados para estruturar a Teoria dos Ciclos Reais de Negócios (\textit{Real Business Cycle Theory, RBC}), com os trabalhos seminais de \textcite{kydland_time_1982} e \textcite{prescott_theory_1986}, \textcite{gali_monetary_2015}.
		\item As principais características dos modelos RBC são: eficiência dos ciclos de negócios; importância dos choques de tecnologia como fontes de flutuações; papel limitado dos fatores monetários.
	\end{itemize}
	
\end{frame}

%%%%%%%%%%%%%%%%%%%%%%%%%%%%%%%%%%%%%%%%%%%%%%%%%%%%%%%%%%%%%%%
%%%%%%%%%%%%%%%%%%%%%%%%%%%%%%%%%%%%%%%%%%%%%%%%%%%%%%%%%%%%%%%

\begin{frame}{New Keynesian Theory}
	
	\begin{itemize}
		\item Em paralelo aos modelos RBC, surgiram os modelos Novos Keynesianos (\textit{New Keynesian Theory, NK}), que procuram dar microfundamentos aos conceitos Keynesianos, \textcite[p.26]{gali_macroeconomic_2007}.
		\item As características de destaque dos modelos NK são: competição monopolística; rigidez nominal de preços; não-neutralidade da moeda no curto prazo.
	\end{itemize}
	
\end{frame}

%%%%%%%%%%%%%%%%%%%%%%%%%%%%%%%%%%%%%%%%%%%%%%%%%%%%%%%%%%%%%%%
%%%%%%%%%%%%%%%%%%%%%%%%%%%%%%%%%%%%%%%%%%%%%%%%%%%%%%%%%%%%%%%
	
\begin{frame}[fragile]{Título Provisório}

	\begin{itemize}
		\item Política Monetária e Economia Regional: Um Modelo Estrutural para Análise dos Impactos da Taxa de Juros sobre uma Região Brasileira.
	\end{itemize}
		
\end{frame}
	
%%%%%%%%%%%%%%%%%%%%%%%%%%%%%%%%%%%%%%%%%%%%%%%%%%%%%%%%%%%%%%%
%%%%%%%%%%%%%%%%%%%%%%%%%%%%%%%%%%%%%%%%%%%%%%%%%%%%%%%%%%%%%%%
	
\section{Referencial Teórico}
	
%%%%%%%%%%%%%%%%%%%%%%%%%%%%%%%%%%%%%%%%%%%%%%%%%%%%%%%%%%%%%%%
%%%%%%%%%%%%%%%%%%%%%%%%%%%%%%%%%%%%%%%%%%%%%%%%%%%%%%%%%%%%%%%

\begin{frame}[fragile]{Modelagem Macroeconômica}
	
	\begin{itemize}
		\item \textit{se você possui uma ideia econômica coesa, você pode colocar em termos de um modelo DSGE} --- \textcite{solis-garcia_ucb_2022}
	\end{itemize}
	
\end{frame}

%%%%%%%%%%%%%%%%%%%%%%%%%%%%%%%%%%%%%%%%%%%%%%%%%%%%%%%%%%%%%%%
%%%%%%%%%%%%%%%%%%%%%%%%%%%%%%%%%%%%%%%%%%%%%%%%%%%%%%%%%%%%%%%

\begin{frame}[fragile]{Modelagem Macroeconômica}
Exemplos de temas:	
	\begin{itemize}
		\item \textcite{pereira_desmatamento_2013}: desmatamento.
		
		\item \textcite{albuquerquemello_mercado_2018}: mercado imobiliário;

		\item \textcite{ribeiro_alongamento_2023}: mercado de trabalho;

	\end{itemize}
	
\end{frame}

%%%%%%%%%%%%%%%%%%%%%%%%%%%%%%%%%%%%%%%%%%%%%%%%%%%%%%%%%%%%%%%
%%%%%%%%%%%%%%%%%%%%%%%%%%%%%%%%%%%%%%%%%%%%%%%%%%%%%%%%%%%%%%%

\begin{frame}[fragile]{Referencial Teórico}
	
	\begin{itemize}
		\item \textcite{costa_junior_understanding_2016}: inicia com RBC e depois adiciona os elementos de NK;
		\item \textcite{gali_monetary_2015}, idem;
		\item \textcite{bergholt_basic_2012}, modelo NK e a programação no \texttt{Dynare};
		\item \textcite{smets_estimated_2003}: modelo para avaliar choques na zona do Euro.
		\item \textcite{smets_shocks_2007}: modelo para avaliar choques nos EUA.
	\end{itemize}
	
\end{frame}

%%%%%%%%%%%%%%%%%%%%%%%%%%%%%%%%%%%%%%%%%%%%%%%%%%%%%%%%%%%%%%%
%%%%%%%%%%%%%%%%%%%%%%%%%%%%%%%%%%%%%%%%%%%%%%%%%%%%%%%%%%%%%%%

\begin{frame}[fragile]{Modelagem Macroeconômica Regionalizada}

\begin{itemize}
	\item \textcite{rickman_modern_2010}: ligação entre a modelagem macroeconômica e a modelagem regional (modelo insumo-produto de
	Leontief, o modelo Walrasiano de equilíbrio geral aplicado e o sistema de equações macroeconométricas).

	\item \textcite{mora_fdi_2019}: efeitos do investimento estrangeiro direto (IED), levando em consideração onde ele é aplicado: modelo estrutural com duas regiões: Bogotá e o resto da Colômbia.

	\item \textcite{costa_junior_dsge_2022}: efeitos da política fiscal, considerando os entes federativos: modelo para o Estados de Goiás e o resto do país.
\end{itemize}
		
\end{frame}

%%%%%%%%%%%%%%%%%%%%%%%%%%%%%%%%%%%%%%%%%%%%%%%%%%%%%%%%%%%%%%%
%%%%%%%%%%%%%%%%%%%%%%%%%%%%%%%%%%%%%%%%%%%%%%%%%%%%%%%%%%%%%%%

\begin{frame}[fragile]{Modelagem Macroeconômica Regionalizada}
	
	\begin{itemize}
		\item \textcite{osterno_uma_2022}: regionalização do SAMBA: SAMBA+REG (\textit{Stochastic Analytical Model with Bayesian Approach} do Banco Central do Brasil).
	\end{itemize}
	
\end{frame}

%%%%%%%%%%%%%%%%%%%%%%%%%%%%%%%%%%%%%%%%%%%%%%%%%%%%%%%%%%%%%%%
%%%%%%%%%%%%%%%%%%%%%%%%%%%%%%%%%%%%%%%%%%%%%%%%%%%%%%%%%%%%%%%

\section{Modelo}
	
%%%%%%%%%%%%%%%%%%%%%%%%%%%%%%%%%%%%%%%%%%%%%%%%%%%%%%%%%%%%%%%
%%%%%%%%%%%%%%%%%%%%%%%%%%%%%%%%%%%%%%%%%%%%%%%%%%%%%%%%%%%%%%%
	
\begin{frame}{Agentes}

O modelo terá quatro agentes:
\begin{itemize}
	\item uma família representativa;
	
	\item firmas produtoras de bens intermediários.
	
	\item uma firma representativa produtora do bem final consumido pelas famílias.

	\item uma autoridade monetária.
\end{itemize}		

\end{frame}

\begin{frame}{Características}

Além disso, também teremos características específicas:

\begin{itemize}
	\item regra de \textcite{calvo_staggered_1983}: gerar fricções nominais nos preços dos bens, alterando as relações de equilíbrio do sistema, gerando a não-neutralidade da moeda no curto prazo, \textcite[p.191]{costa_junior_understanding_2016}.
	
	\item os choques estocásticos estarão presentes na produtividade das firmas e nas preferências da família representativa.
	
	\item regionalização do modelo: um índice para a região estudada e o restante do Brasil, de tal forma que teremos as famílias, a firma de bens finais e as firmas de bens intermediários de cada região.
	
	\item as famílias não terão mobilidade, mas os bens intermediários e finais terão, e esse será o elo para conectar as duas regiões.
\end{itemize}

\end{frame}
	
%%%%%%%%%%%%%%%%%%%%%%%%%%%%%%%%%%%%%%%%%%%%%%%%%%%%%%%%%%%%%%%
%%%%%%%%%%%%%%%%%%%%%%%%%%%%%%%%%%%%%%%%%%%%%%%%%%%%%%%%%%%%%%%	
	
%\begin{itemize}
%	\item Em relação aos objetivos \cite{gil2002elaborar}:
%	\begin{itemize}
%		\item Pesquisa Explicativa: investigar a transmissão dos choques de política monetária para uma economia regional.
%	\end{itemize}
%	\item Em relação aos procedimentos \cite{gil2002elaborar}:
%	\begin{itemize}
%		\item Pesquisa Bibliográfica: explorar e conhecer a teoria e os modelos existentes;
%		\item Pesquisa Experimental: utilizar dados reais para verificar a capacidade preditiva do modelo.
%	\end{itemize}
%\end{itemize}
		

	
%%%%%%%%%%%%%%%%%%%%%%%%%%%%%%%%%%%%%%%%%%%%%%%%%%%%%%%%%%%%%%%
%%%%%%%%%%%%%%%%%%%%%%%%%%%%%%%%%%%%%%%%%%%%%%%%%%%%%%%%%%%%%%%
	
\section{Resultados Esperados}
	
%%%%%%%%%%%%%%%%%%%%%%%%%%%%%%%%%%%%%%%%%%%%%%%%%%%%%%%%%%%%%%%
%%%%%%%%%%%%%%%%%%%%%%%%%%%%%%%%%%%%%%%%%%%%%%%%%%%%%%%%%%%%%%%
	
\begin{frame}[fragile]{Resultados Esperados}
		
		\begin{itemize}
			\item Esperamos que o modelo demonstre que uma região brasileira responde um choque de política monetária, gerando uma variação no produto regional.
			\item Por exemplo, um choque de $1\%$ na taxa de juros gera uma diminuição de $x\%$ do produto de um Estado brasileiro.
		\end{itemize}
		
	\end{frame}
	
	%%%%%%%%%%%%%%%%%%%%%%%%%%%%%%%%%%%%%%%%%%%%%%%%%%%%%%%%%%%%%%%
	%%%%%%%%%%%%%%%%%%%%%%%%%%%%%%%%%%%%%%%%%%%%%%%%%%%%%%%%%%%%%%%
	
	\section{Cronograma}
	
	%%%%%%%%%%%%%%%%%%%%%%%%%%%%%%%%%%%%%%%%%%%%%%%%%%%%%%%%%%%%%%%
	%%%%%%%%%%%%%%%%%%%%%%%%%%%%%%%%%%%%%%%%%%%%%%%%%%%%%%%%%%%%%%%
	
\begin{frame}[fragile]{Cronograma}
		
% Please add the following required packages to your document preamble:
% \usepackage{graphicx}
\renewcommand{\arraystretch}{1.5}
\begin{table}[]
\tiny
%\centering
\resizebox{\textwidth}{!}{%
\begin{tabular}{l|ccccccccc}
\hline
Atividade              &mar&abr&mai&jun&jul&ago&set&out&nov\\ \hline
Pesquisa Bibliográfica & x & x & x &   &   &   &   &   &   \\ \hline
Projeto de Pesquisa    &   &   & x &   &   &   &   &   &   \\ \hline
Modelagem              &   &   & x & x &   &   &   &   &   \\ \hline
Programação no
\texttt{Dynare}        &   &   &   & x &   &   &   &   &   \\ \hline
Seminário do Projeto   &   &   &   & x &   &   &   &   &   \\ \hline
Coleta dos dados       &   &   &   &   & x &   &   &   &   \\ \hline
Tratamento dos dados   &   &   &   &   & x &   &   &   &   \\ \hline
Cálculo dos Parâmetros &   &   &   &   &   & x &   &   &   \\ \hline
Banca de Qualificação  &   &   &   &   &   & x &   &   &   \\ \hline
Análise dos Resultados &   &   &   &   &   &   & x &   &   \\ \hline
Revisão e Edição       &   &   &   &   &   &   &   & x &   \\ \hline
Defesa da Dissertação  &   &   &   &   &   &   &   &   & x \\ \hline
Reuniões de Orientação & x & x & x & x & x & x & x & x & x \\ \hline
\end{tabular}%
}
\end{table}
\renewcommand{\arraystretch}{1}
\end{frame}
	
%%%%%%%%%%%%%%%%%%%%%%%%%%%%%%%%%%%%%%%%%%%%%%%%%%%%%%%%%%%%%%%
%%%%%%%%%%%%%%%%%%%%%%%%%%%%%%%%%%%%%%%%%%%%%%%%%%%%%%%%%%%%%%%
	
\begin{frame}[allowframebreaks]{Referências Iniciais}

	\printbibliography[title={Referências Iniciais}]

\end{frame}
	
%%%%%%%%%%%%%%%%%%%%%%%%%%%%%%%%%%%%%%%%%%%%%%%%%%%%%%%%%%%%%%%
%%%%%%%%%%%%%%%%%%%%%%%%%%%%%%%%%%%%%%%%%%%%%%%%%%%%%%%%%%%%%%%
	
{\setbeamercolor{palette primary}{fg=black, bg=white}
	\begin{frame}[standout]
	Dúvidas e Sugestões
	\end{frame}
}
	
	%%%%%%%%%%%%%%%%%%%%%%%%%%%%%%%%%%%%%%%%%%%%%%%%%%%%%%%%%%%%%%%
	%%%%%%%%%%%%%%%%%%%%%%%%%%%%%%%%%%%%%%%%%%%%%%%%%%%%%%%%%%%%%%%
	
	{\setbeamercolor{palette primary}{fg=black, bg=white}
		\begin{frame}[standout]
			Obrigado! \\
			andreluizmtg@gmail.com\\
			41.98460.2209
		\end{frame}
	}
	
	%%%%%%%%%%%%%%%%%%%%%%%%%%%%%%%%%%%%%%%%%%%%%%%%%%%%%%%%%%%%%%%
	%%%%%%%%%%%%%%%%%%%%%%%%%%%%%%%%%%%%%%%%%%%%%%%%%%%%%%%%%%%%%%%
	
\end{document}