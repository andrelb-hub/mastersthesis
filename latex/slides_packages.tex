% --------------------------------------------------
% BEAMER METROPOLIS
% --------------------------------------------------

\usetheme[progressbar=frametitle]{metropolis}

\usepackage{appendixnumberbeamer}

\usepackage{booktabs}

\usepackage[scale=2]{ccicons}

\usepackage{pgfplots}

\usepgfplotslibrary{dateplot}

\usepackage{xspace}

\newcommand{\themename}{\textbf{\textsc{metropolis}}\xspace}

% --------------------------------------------------
% MY PACKAGES
% --------------------------------------------------

\usepackage{setspace}
%\singlespacing
\onehalfspacing
%\doublespacing

% Fira Font:
\usepackage[sfdefault]{FiraSans}

% cor da barra de progresso:
\setbeamercolor{progress bar}{fg=orange,bg=white}

% largura da barra de progresso:
\makeatletter
\setlength{\metropolis@titleseparator@linewidth}{2pt}
\setlength{\metropolis@progressonsectionpage@linewidth}{2pt}
\setlength{\metropolis@progressinheadfoot@linewidth}{2pt}
\makeatother

% itens transparentes:
\setbeamercovered{transparent}

% space between items:
\newlength{\wideitemsep}
\setlength{\wideitemsep}{\itemsep}\addtolength{\wideitemsep}{10pt}

% --------------------------------------------------
% SPACES BETWEEN ITEMS
% --------------------------------------------------

% reconfigure itemize lists:
\let\olditem\item
\renewcommand{\item}{%
	\olditem\vspace{5pt}}

% --------------------------------------------------
% MULTIFIGURES AND SUBFIGURES
% --------------------------------------------------

\usepackage{caption}
\usepackage{subcaption}

% --------------------------------------------------
% MATH COMMANDS
% --------------------------------------------------

% differential d:
\DeclareMathOperator{\dif}{d}

% subject to:
\DeclareMathOperator{\st}{s.t.}

% expectation symbol:
\newcommand{\E}[1][t]{{\mathbb{E}_{#1}}}

% subscript text: $_t$
\newcommand{\subt}[1][t]{{_{#1}}}

% superscript text: $^$
\newcommand{\supt}[1]{{^{#1}}}

% the greek letter omicron:
\newcommand{\omicron}{o}

% --------------------------------------------------
% OTHER COMMANDS
% --------------------------------------------------

% the name of the command:
\newcommand{\com}[1]{\texttt{\textbackslash #1}}

% the name of the software:
% \newcommand{\dynare}{\texttt{Dynare}}
\newcommand{\dynare}{\colorbox{lightgray}{$\mathsf{Dynare}$}}
\newcommand{\matlab}{\texttt{MATLAB}}
\newcommand{\octave}{\texttt{OCTAVE}}

% checkmark and xmark:
% usage: \cmark, \xmark
\usepackage{pifont}% http://ctan.org/pkg/pifont
\newcommand{\cmark}{\textcolor{teal}{\ding{51}}}
\newcommand{\xmark}{\textcolor{gray}{\ding{55}}}

% --------------------------------------------------
% COMMENT ENVIRONMENT
% --------------------------------------------------

\usepackage{comment}