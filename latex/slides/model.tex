% --------------------------------------------------
% DOCUMENT CLASS
% --------------------------------------------------

\documentclass[
presentation.tex
]{subfiles}

\begin{document}

\section{Model}
	
% --------------------------------------------------
% SLIDE
% --------------------------------------------------
	
	\begin{frame}{Agents}

	The model has four agents:
	\begin{itemize}
	
	\item the representative household maximizes utility;
	
	\item firms producing intermediate goods minimize costs and maximize profit flow;
	
	\item firms producing final goods maximize profit.

	\item the monetary authority determines the interest rate, aiming to control inflation and pursuing economic growth.
	
	\end{itemize}		

	\end{frame}

% --------------------------------------------------
% SLIDE
% --------------------------------------------------

	\begin{frame}{Household}
		
		
		
		
	\end{frame}

% --------------------------------------------------
% SLIDE
% --------------------------------------------------

\begin{frame}{Características}

Além disso, também teremos características específicas:

\begin{itemize}
	\item regra de \textcite{calvo_staggered_1983}: gerar fricções nominais nos preços dos bens, alterando as relações de equilíbrio do sistema, gerando a não-neutralidade da moeda no curto prazo, \textcite[p.191]{costa_junior_understanding_2016}.
	
	\item os choques estocásticos estarão presentes na produtividade das firmas e nas preferências da família representativa.
	
	\item regionalização do modelo: um índice para a região estudada e o restante do Brasil, de tal forma que teremos as famílias, a firma de bens finais e as firmas de bens intermediários de cada região.
	
	\item as famílias não terão mobilidade, mas os bens intermediários e finais terão, e esse será o elo para conectar as duas regiões.
\end{itemize}

\end{frame}	

\end{document}