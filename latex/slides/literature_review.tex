% --------------------------------------------------
% DOCUMENT CLASS
% --------------------------------------------------

\documentclass[presentation.tex]{subfiles}

\begin{document}

\section{Literature Review}

% --------------------------------------------------
% SLIDE
% --------------------------------------------------

\subsection{Macro modeling}

\begin{frame}[fragile]{Macro modeling}
	
	\begin{itemize}
		
		\item \textcite{costa_junior_understanding_2016}: presents a RBC model and then adds NK elements in each chapter;
		
		\item \textcite{gali_monetary_2015}: discuss monetary policy starting with a RBC model and also adds NK elements in each chapter;
		
		\item \textcite{bergholt_basic_2012}: presents a NK and the method of programming in \texttt{Dynare};
		
		\item \textcite{solis-garcia_ucb_2022}: presents a RBC model and demonstrate the math tools necessary to solve a DSGE model;
				
	\end{itemize}
	
\end{frame}

% --------------------------------------------------
% SLIDE
% --------------------------------------------------

\begin{frame}[fragile]{Regional Modeling}

\begin{itemize}

	\item \textcite{rickman_modern_2010}: link between macro and regional modeling.

	\item \textcite{mora_fdi_2019}: efeitos do investimento estrangeiro direto (IED), levando em consideração onde ele é aplicado: modelo estrutural com duas regiões: Bogotá e o resto da Colômbia.

	\item \textcite{costa_junior_dsge_2022}: efeitos da política fiscal, considerando os entes federativos: modelo para o Estados de Goiás e o resto do país.
	
\end{itemize}
		
\end{frame}

% --------------------------------------------------
% SLIDE
% --------------------------------------------------

\begin{frame}{Modelagem Macroeconômica Regionalizada}
	
	\begin{itemize}
		
		\item \textcite{osterno_uma_2022}: regionalização do SAMBA: SAMBA+REG (\textit{Stochastic Analytical Model with Bayesian Approach} do Banco Central do Brasil).
				
	\end{itemize}
	
\end{frame}

\end{document}