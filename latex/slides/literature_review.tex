% --------------------------------------------------
% DOCUMENT CLASS
% --------------------------------------------------

\documentclass[presentation.tex]{subfiles}

\begin{document}

\section{Literature Review}

% --------------------------------------------------
% SLIDE
% --------------------------------------------------
	
\begin{frame}[fragile]{Modelagem Macroeconômica}
	
	\begin{itemize}
		\item \textit{se você possui uma ideia econômica coesa, você pode colocar em termos de um modelo DSGE} --- \textcite{solis-garcia_ucb_2022}
	\end{itemize}
	
\end{frame}

% --------------------------------------------------
% SLIDE
% --------------------------------------------------

\begin{frame}[fragile]{Modelagem Macroeconômica}
	
Exemplos de temas:	
	\begin{itemize}
		
		\item \textcite{pereira_desmatamento_2013}: desmatamento.
		
		\item \textcite{albuquerquemello_mercado_2018}: mercado imobiliário;

		\item \textcite{ribeiro_alongamento_2023}: mercado de trabalho;

	\end{itemize}
	
\end{frame}

% --------------------------------------------------
% SLIDE
% --------------------------------------------------

\begin{frame}[fragile]{Referencial Teórico}
	
	\begin{itemize}
		
		\item \textcite{costa_junior_understanding_2016}: inicia com RBC e depois adiciona os elementos de NK;
		
		\item \textcite{gali_monetary_2015}, idem;
		
		\item \textcite{bergholt_basic_2012}, modelo NK e a programação no \texttt{Dynare};
		
		\item \textcite{smets_estimated_2003}: modelo para avaliar choques na zona do Euro.
		
		\item \textcite{smets_shocks_2007}: modelo para avaliar choques nos EUA.
		
	\end{itemize}
	
\end{frame}

% --------------------------------------------------
% SLIDE
% --------------------------------------------------

\begin{frame}[fragile]{Modelagem Macroeconômica Regionalizada}

\begin{itemize}

	\item \textcite{rickman_modern_2010}: ligação entre a modelagem macroeconômica e a modelagem regional (modelo insumo-produto de
	Leontief, o modelo Walrasiano de equilíbrio geral aplicado e o sistema de equações macroeconométricas).

	\item \textcite{mora_fdi_2019}: efeitos do investimento estrangeiro direto (IED), levando em consideração onde ele é aplicado: modelo estrutural com duas regiões: Bogotá e o resto da Colômbia.

	\item \textcite{costa_junior_dsge_2022}: efeitos da política fiscal, considerando os entes federativos: modelo para o Estados de Goiás e o resto do país.
	
\end{itemize}
		
\end{frame}

% --------------------------------------------------
% SLIDE
% --------------------------------------------------

\begin{frame}{Modelagem Macroeconômica Regionalizada}
	
	\begin{itemize}
		
		\item \textcite{osterno_uma_2022}: regionalização do SAMBA: SAMBA+REG (\textit{Stochastic Analytical Model with Bayesian Approach} do Banco Central do Brasil).
				
	\end{itemize}
	
\end{frame}

\end{document}