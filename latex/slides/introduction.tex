% --------------------------------------------------
% DOCUMENT CLASS
% --------------------------------------------------

\documentclass[
presentation.tex
]{subfiles}

\begin{document}
	
\section{Introduction}

% --------------------------------------------------
% SLIDE
% --------------------------------------------------

\begin{frame}{Introduction}
		
\begin{itemize}
	\item A modelagem macroeconômica é uma importante ferramenta para estudar as ligações entre a economia monetária e os resultados dos agregados de um país, \textcite{gali_monetary_2015}.
	\item As regiões brasileiras possuem matrizes e setores econômicos heterogêneos que respondem de diferentes formas às decisões da autoridade monetária, \textcite{bertanha_efeitos_2008}.
\end{itemize}

\end{frame}

% --------------------------------------------------
% SLIDE
% --------------------------------------------------

\begin{frame}{Introduction}
	
	\begin{itemize}
		\item \textit{Na realidade, a maior parte das tolices já escritas e que se continuam a escrever sobre economia poderia ter sido poupada se todo economista fosse obrigado a expor suas ideias construindo um modelo matemático} --- \textcite[p.68]{simonsen_microeconomia_1979}.
	\end{itemize}
	
\end{frame}

% --------------------------------------------------
% SLIDE
% --------------------------------------------------

\begin{frame}{Introduction}
	
	\begin{itemize}
		\item Proposta: desenvolver um modelo estrutural com desdobramentos regionais, utilizando a metodologia DSGE (\textit{Dynamic and Stochastic General Equilibrium}).
		\item Objetivo: verificar se existe correlação entre a taxa de juros nominal da economia (uma variável macroeconômica) e o nível de produção de uma região brasileira (uma variável regional). Existindo esta correlação, pretendemos quantificá-la.
	\end{itemize}
	
\end{frame}

% --------------------------------------------------
% SLIDE
% --------------------------------------------------

\begin{frame}{O que é um modelo DSGE?}
	
	\begin{itemize}
		\item Os modelos DSGE são ferramentas utilizadas em macroeconomia para avaliar a relação existente entre as variáveis selecionadas pelo pesquisador.
		\item Tem como principais características um horizonte de tempo infinito e choques aleatórios sobre algumas variáveis de interesse.
	\end{itemize}
	
\end{frame}

% --------------------------------------------------
% SLIDE
% --------------------------------------------------

\begin{frame}{Real Business Cycles Theory}
	
	\begin{itemize}
		\item Os modelos DSGE começaram a ser usados para estruturar a Teoria dos Ciclos Reais de Negócios (\textit{Real Business Cycle Theory, RBC}), com os trabalhos seminais de \textcite{kydland_time_1982} e \textcite{prescott_theory_1986}, \textcite{gali_monetary_2015}.
		\item As principais características dos modelos RBC são: eficiência dos ciclos de negócios; importância dos choques de tecnologia como fontes de flutuações; papel limitado dos fatores monetários.
	\end{itemize}
	
\end{frame}

% --------------------------------------------------
% SLIDE
% --------------------------------------------------

\begin{frame}{New Keynesian Theory}
	
	\begin{itemize}
		\item Em paralelo aos modelos RBC, surgiram os modelos Novos Keynesianos (\textit{New Keynesian Theory, NK}), que procuram dar microfundamentos aos conceitos Keynesianos, \textcite[p.26]{gali_macroeconomic_2007}.
		\item As características de destaque dos modelos NK são: competição monopolística; rigidez nominal de preços; não-neutralidade da moeda no curto prazo.
	\end{itemize}
	
\end{frame}

\end{document}