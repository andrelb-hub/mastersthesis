% --------------------------------------------------
% DOCUMENT CLASS
% --------------------------------------------------

\documentclass[
thesis.tex
]{subfiles}

\begin{document}

% --------------------------------------------------
% APPENDIX
% --------------------------------------------------

\section{Appendix}

% --------------------------------------------------
% TABLE OF PARAMETERS
% --------------------------------------------------

\subsection{Greek Letters}

\vspace*{0.5cm}

\begin{center}
	
	\begin{longtblr}[
		label = {table:greek-letters},
		caption = {Greek Letters},
		remark{Source} = {The Author.},
		]{rowhead = 1,
		colspec = {c|l|l}}
		\hline[2pt]
		Parameter     & Command         & Definition                   \\ \hline[2pt]
		$\alpha$      &\com{alpha}      & factor relative weight in production \\
		$\beta$       &\com{beta}       & intertemporal discount       \\
		$\gamma$      &\com{gamma}      & interest rate sensitivity    \\
		$\delta$      &\com{delta}      & depreciation rate            \\
		$\epsilon$    &\com{epsilon}    & \\
		$\varepsilon$ &\com{varepsilon} & stochastic error            \\
		$\zeta$       &\com{zeta}       & \\
		$\eta$        &\com{eta}        & household region (destination) \\
		$\theta$      &\com{theta}      & price rigidity level        \\
		$\vartheta$   &\com{vartheta}   & regional inflation relative weight in total inflation \\
		$\iota$       &\com{iota}       & \\
		$\kappa$      &\com{kappa}      & \\
		$\varkappa$   &\com{varkappa}   & \\
		$\lambda$     &\com{lambda}     & real marginal cost          \\
		$\Lambda$     &\com{Lambda}     & nominal marginal cost       \\
		$\mu$         &\com{mu}         & household Lagrangian multiplier \\
		$\nu$         &\com{nu}         & firm and goods region (origin) \\
		$\xi$         &\com{xi}         & \\
		$\omicron$    &\com{omicron}    & \\
		$\pi$         &\com{pi}         & inflation                   \\
		$\varpi$      &\com{varpi}      & \\
		$\rho$        &\com{rho}        & autoregressive parameter    \\
		$\varrho$     &\com{varrho}     & nominal discount rate in steady state \\
		$\sigma$      &\com{sigma}      & relative risk aversion      \\
		$\varsigma$   &\com{varsigma}   & \\
		$\tau$        &\com{tau}        & \\
		$\upsilon$    &\com{upsilon}    & \\
		$\phi$        &\com{phi}        & labor relative weight in utility \\
		$\varphi$     &\com{varphi}     & marginal disutility of labor supply\\
		$\chi$        &\com{chi}        & \\
		$\psi$        &\com{psi}        & elasticity of substitution between intermediate goods \\
		$\omega$      &\com{omega}      & consumption relative weight in consumption bundle \\
		\hline[2pt]
	\end{longtblr}
	
\end{center}
	
% --------------------------------------------------
% LITERATURE REVIEW
% --------------------------------------------------

\subsection{Table of the Literature Review}

A table of the literature review will be presented here, in order to compare the elements of each DSGE model discussed in the text.

% --------------------------------------------------
% DEFINITIONS
% --------------------------------------------------

\subsection{Definitions, Theorems and Lemmas}

The objective of this appendix is to present the definitions, theorems, lemmas and proofs used throughout the text.

\subsubsection*{Household}

\begin{comment}
	
	% --------------------------------------------------
	% Household Maximization Problem
	% --------------------------------------------------
	
	\begin{definition}[Household Maximization Problem]
		{\singlespacing
			The utility function is:
			\begin{itemize}
				\item strictly increasing in consumption $C$;
				\item strictly increasing in leisure $l$;
				\item strictly concave;
				\item twice continuously differentiable;
				\item the composite consumption good $C$ is also the numeraire good, so that its price equals one: $p_C=1$;
				\item to avoid corner solutions, the Inada conditions\footnotemark{} hold. \footnotetext{see definition \ref{def:Inada Condition}.}
		\end{itemize}}
		
		Consider a representative household that maximizes an utility function $u$ that depends on consumption $C_t$ and labor $L_t$:
		\begin{align}
			u \equiv u \left( C_t, L_t \right)
		\end{align}
		
		The utility function is considered to be convex (when a variable increases, the respective marginal utility diminishes)\footnotemark{}: \footnotetext{Consider the following notation: given two variables $X$ and $Y$, the first and second partial derivatives are: $Y_X := \displaystyle\frac{\partial Y}{\partial X}$ and $Y_{XX} := \displaystyle\frac{\partial^2 Y}{\partial X^2}$.}
		\begin{align*}
			u_{C} > 0 \text{,}\quad u_{CC} < 0 \text{,}\quad
			u_{L} > 0 \text{,}\quad u_{LL} < 0
		\end{align*}
		
	\end{definition}
	
\end{comment}

\begin{definition}[Discount Factor $\beta$]
	
	Other things the same, a unit of consumption enjoyed tomorrow is less valuable (yields less utility) than a unit of consumption enjoyed today \cite[Lecture 2, p.1]{solis-garcia_ucb_2022}.
	
\end{definition}

% --------------------------------------------------
% Inada Condition
% --------------------------------------------------

\begin{definition}[Inada Condition] \label{def:Inada Condition}
	The Inada conditions \cite{inada_two-sector_1963} avoid corner solutions. For this purpose, it is assumed that the partial derivatives $u_C$ and $u_L$ of the function $u(C, L)$ satisfy the following rules:
	\begin{align}
		\lim_{C\to 0} u_C(C,L^\ast) = \infty \quad \text{and} \quad
		\lim_{C\to \infty} u_C(C,L^\ast) = 0 \\
		\lim_{L\to 0} u_C(C^\ast,L) = \infty \quad \text{and} \quad
		\lim_{L\to \infty} u_C(C^\ast,L) = 0 \nonumber
	\end{align}
	where $C^\ast,L^\ast \in \mathbb{R}_{++}$ and $u_j$ is the partial derivative of the utility function with respect to $j=C,L$ \cite[Lecture 1, p.2]{solis-garcia_ucb_2022}
\end{definition}

% --------------------------------------------------
% Transversality Condition
% --------------------------------------------------

\begin{definition}[Transversality Condition]
	\cite[Lecture 4, p.4]{solis-garcia_ucb_2022}
\end{definition}

% --------------------------------------------------
% FIRMS
% --------------------------------------------------

\subsubsection*{Firms}

% --------------------------------------------------
% MARGINAL COST
% --------------------------------------------------

\begin{lemma}[Marginal Cost]\label{lemma:marginal-cost}
	The Lagrangian multiplier $\Lambda_t$ is the nominal marginal cost of the intermediate-good firm:
	\begin{align}
		MC_t \coloneq \frac{\partial TC_t}{\partial Y_t} = \Lambda_t
	\end{align}
	
	\begin{proof}
		\textcite[p.449]{simon_mathematics_1994}.
	\end{proof}
	
\end{lemma}

% --------------------------------------------------
% Constant Returns to Scale
% --------------------------------------------------

\begin{definition}[Constant Returns to Scale]
	\cite[Lecture 1, p.5]{solis-garcia_ucb_2022}
\end{definition}

\begin{definition}[Homogeneous Function of Degree $k$]
	\cite[Lecture 1, p.5]{solis-garcia_ucb_2022}
\end{definition}

\subsubsection*{Monetary Authority}

\subsubsection*{Shocks}

\subsubsection*{Equilibrium Conditions}

\begin{definition}[Competitive Equilibrium]
	\cite[Lecture 1, p.6]{solis-garcia_ucb_2022}
\end{definition}

% --------------------------------------------------
% STEADY STATE
% --------------------------------------------------

\subsubsection*{Steady State}

% --------------------------------------------------
% INFLATION LEMMA
% --------------------------------------------------

\begin{lemma}[Steady State Inflation]\label{lemma:steady-state-inflation}
	
	In steady state, prices are stable $P_t = P_{t-1} = P$ and the gross inflation rate is one.
	\begin{proof} Equation \ref{eq:ss-gross-inflation-rate}. \end{proof}  \end{lemma}

\begin{corollary}\label{coro:steady-state-YKL}
	
	In steady state, all firms have the same level of production $Y$ and therefore demand the same amount of factors, capital $K$ and labor $L$.
	\begin{align*}
		P_t = P_{t-1} = P \implies 
		\begin{pmatrix}
			Y_j & K_j & L_j
		\end{pmatrix} =
		\begin{pmatrix}
			Y & K & L
		\end{pmatrix}
	\end{align*}
	
\end{corollary}

% --------------------------------------------------
% LOG-LINEARIZATION
% --------------------------------------------------

\subsubsection*{Log-linearization}

% --------------------------------------------------
% PERCENTAGE DEVIATION
% --------------------------------------------------

\begin{definition}[PERCENTAGE DEVIATION]\label{def:percentage-deviation}
	
	The percentage deviation of a variable $x_t$ from its steady state is given by \cite[Lecture 6, p.2]{solis-garcia_ucb_2022}:
	\begin{align}
		\hat{x}_t \coloneq \frac{x_t - x}{x} \label{eq:percentage-deviation}
	\end{align}
	
\end{definition}

% --------------------------------------------------
% UHLIG'S RULES
% --------------------------------------------------

\begin{lemma}[UHLIG'S RULES]\label{lemma:uhligs-rules}
	
	The Uhlig's rules are a set of approximations used to log-linearize equations \cite[Lecture 6, p.2]{solis-garcia_ucb_2022}.
	
	\begin{itemize}
		\item Rule 1: \label{uhlig-rule-1}
		
		\( x_t = x(1 + \hat{x}_t) \) 
		
		\item Rule 2 (Product):
		
		
		
		\item Rule 3 (Exponential):
		
		
	\end{itemize}
	
\end{lemma}

\begin{corollary}[Logarithm Rule]\label{coro:logarithm-rule}
	
	\begin{align*}
		\ln x_t \approx \ln x + \hat{x}_t
	\end{align*}
	
\end{corollary}

% --------------------------------------------------
% LEVEL DEVIATION
% --------------------------------------------------

\begin{definition}[LEVEL DEVIATION]\label{def:level-deviation}
	
	The level deviation of a variable $u_t$ from its steady state is given by: \cite[Lecture 9, p.9]{solis-garcia_ucb_2022}
	\begin{align}
		\widetilde{u}_t \coloneq u_t - u \label{eq:level-deviation}
	\end{align}
	
\end{definition}

% --------------------------------------------------
% UHLIG'S RULES FOR LEVEL DEVIATIONS
% --------------------------------------------------

\begin{lemma}[UHLIG'S RULES FOR LEVEL DEVIATIONS]\label{lemma:level-rules}
	
	Uhlig's rules can be applied to level deviations in order to log-linearize equations \cite[Lecture 6, p.2]{solis-garcia_ucb_2022}.
	
	\begin{itemize}
		\item Rule 1:
		\begin{align}
			\label{lemma:level-rule-1a}
			u_t &= u + \widetilde{u}_t \\
			\label{lemma:level-rule-1b}
			u_t &= u\left(1+ \frac{\widetilde{u}_t}{u} \right)
		\end{align}
		
		\item Rule 2 (Product):
		
		\item Rule 3 (Exponential):
		
		\item Rule 4 (Logarithm):
		
		\item Rule 5 (Percentage and Level Deviations)
		
	\end{itemize}
	
\end{lemma}

% --------------------------------------------------
% PRODUCT OPERATOR
% --------------------------------------------------

\begin{lemma}[LEVEL DEVIATION OF THE PRESENT VALUE DISCOUNT FACTOR]\label{product-operator}
	
	The level deviation of the present value discount factor is equivalent to \cite[Lecture 13, p.6]{solis-garcia_ucb_2022}:
	\begin{align}
		\label{eq:product-operator}
		\prod_{k=0}^{s-1}(1+R_{t+k}) = (1 + R)^s \left( 1 + \frac{1}{1 + R} \sum_{k=0}^{s-1} \widetilde{R}_{t+k} \right)
	\end{align}
	
	\begin{proof}
		Substitute the interest rate by the gross interest rate $GR_t = 1 + R_t$ and apply rule \ref{lemma:level-rule-1b}:
		\begin{align*}
			& \prod_{k=0}^{s-1}(1+R_{t+k}) = \prod_{k=0}^{s-1}(GR_{t+k})
			&\implies \nonumber \\
			& GR \times \dots \times GR \left( 1 + \frac{1}{GR} \widetilde{GR}_t + \frac{1}{GR} \widetilde{GR}_{t+1} + \dots + \frac{1}{GR} \widetilde{GR}_{t+s-1} \right)
			&\implies \nonumber \\
			& GR^s \left( 1 + \frac{1}{GR} \sum_{k=0}^{s-1} \widetilde{GR}_{t+k} \right)
			&\implies \nonumber \\
			& (1 + R)^s \left( 1 + \frac{1}{1 + R} \sum_{k=0}^{s-1} \widetilde{R}_{t+k} \right) &\,
		\end{align*}
	\end{proof}
	
\end{lemma}

% --------------------------------------------------
% PERCENTAGE DIVISION
% --------------------------------------------------

%\begin{lemma}[PERCENTAGE DIVISION]\label{lemma:percentage-division}
%	The division of gross percentages is equivalent to the subtraction of percentages.
%	\begin{align}
	%		\frac{1+x}{1+y} \approx 1 + x - y
	%	\end{align}
%	where \( x,y \in [0,1] \) and \( X,Y \geq 0 \).
%	\begin{proof}
	%		\begin{align}
		%			\frac{1+x}{1+y} = \frac{1 + \frac{X}{100}}{1 + \frac{Y}{100}} = \frac{100 + X}{100} \cdot \frac{100}{100 + Y} = \frac{100 + X}{100 + Y}
		%		\end{align}
	%	\end{proof}
%\end{lemma}

% --------------------------------------------------
% GEOMETRIC SERIES
% --------------------------------------------------

\begin{definition}[Geometric Series]\label{def:geometric-series}
	
	A geometric series is the sum of the terms of a geometric sequence.
	\begin{align*}
		S_\infty = \sum_{i=0}^{\infty} ar^i \implies 
		S_\infty = \frac{a}{1-r} \; , \; |r| <1
	\end{align*}
	
\end{definition}

% --------------------------------------------------
% LAG OPERATOR
% --------------------------------------------------

\begin{definition}[LAG AND LEAD OPERATORS]\label{def:lag-operator}
	The lag operator $\mathbb{L}$ is a mathematical operator that represents the backshift or lag of a time series \cite[Lecture 13, p.9]{solis-garcia_ucb_2022}:
	\begin{align*}
		\mathbb{L} x_t            & = x_{t-1}              \\
		(1 + a\mathbb{L})y_{t+2} & = y_{t+2} + ay_{t+1}
	\end{align*}
	
	Analogously, the lead operator $\mathbb{L}^{-1}$ (or inverse lag operator) yields a variable's lead \cite[Lecture 13, p.9]{solis-garcia_ucb_2022}:
	\begin{align*}
		\mathbb{L}^{-1} x_t            & = x_{t+1}              \\
		(1 + a\mathbb{L}^{-1}) y_{t+2} & = y_{t+2} + ay_{t+3}
	\end{align*}
\end{definition}

\subsubsection*{Canonical NK Model}

% --------------------------------------------------
% DEFINITION
% --------------------------------------------------

\begin{comment}

\begin{definition}[Canonical NK Model]
	
	\cite[Lecture 13, p.7]{solis-garcia_ucb_2022}
	
	3.1.2 Back to the pricing equation:
	
	log-linearize the left hand equation:
	\begin{align*}
		&\mathbb{E}_t \sum_{s=0}^{\infty} 
		\left[ 
		\left( \frac{\theta}{1+R} \right)^s
		\left( \frac{P_t^\ast Y_{t+s}(j)}{1 + \frac{1}{1+R}
			\sum_{k=0}^{s-1} \widetilde{R}_{t+k}} \right) 
		\right]
		\implies \\
		&\mathbb{E}_t \sum_{s=0}^{\infty} 
		\left[ 
		\left( \frac{\theta}{1+R} \right)^s
		\left( \frac{P^\ast Y(j)(1+\widehat{P}_t^\ast + \widehat{Y}_{t+s}(j))}{1 + \frac{1}{1+R}
			\sum_{k=0}^{s-1} \widetilde{R}_{t+k}} \right) 
		\right] \implies \\
		&\mathbb{E}_t \sum_{s=0}^{\infty} 
		\left[ 
		\left( \frac{\theta}{1+R} \right)^s
		\left( \frac{P_t^\ast Y_{t+s}(j)}{\frac{(1+R)+\sum_{k=0}^{s-1} \widetilde{R}_{t+k}}{1+R}} \right) 
		\right]
		\implies \\
		&\mathbb{E}_t \sum_{s=0}^{\infty} 
		\left[ 
		\left( \frac{\theta}{1+R} \right)^s
		\left( \frac{P_t^\ast Y_{t+s}(j)(1+R)}{(1+R)+\sum_{k=0}^{s-1} \widetilde{R}_{t+k}} \right) 
		\right]
		\implies \\
		&\mathbb{E}_t \sum_{s=0}^{\infty} 
		\left[ 
		\left( \frac{\theta}{1+R} \right)^s
		\left( \frac{P^\ast Y(j)(1+\widehat{P}_t^\ast + \widehat{Y}_{t+s}(j))(1+R)}{(1+R)+\sum_{k=0}^{s-1} \widetilde{R}_{t+k}} \right) 
		\right]
	\end{align*}
	
\end{definition}

\end{comment}

\begin{definition}[Medium Scale DSGE Model]
	A Medium Scale DSGE Model has habit formation, capital accumulation, indexation, etc. \cite[p.208]{gali_monetary_2015}. 
	
	See Galí, Smets, and Wouters (2012) for an analysis of the sources of unemployment fluctuations in an estimated medium-scale version of the present model.
\end{definition}

\begin{definition}[Stochastic Process]
	\cite[Lecture 5, p.3]{solis-garcia_ucb_2022}.
\end{definition}

\begin{definition}[Markov Process]
	\cite[Lecture 5, p.4]{solis-garcia_ucb_2022}.
\end{definition}

\begin{definition}[first-order autoregressive process $AR(1)$]
	the first-order autoregressive process $AR(1)$ \cite[Lecture 5, p.4]{solis-garcia_ucb_2022}.
\end{definition}

\begin{definition}[Blanchard-Kahn Conditions]
	\cite[Hands on 5, p.14]{solis-garcia_ucb_2022}.
\end{definition}


% --------------------------------------------------
% DEFINITION
% --------------------------------------------------

% --------------------------------------------------
% DEFINITION
% --------------------------------------------------

% --------------------------------------------------
% DEFINITION
% --------------------------------------------------

% --------------------------------------------------
% DEFINITION
% --------------------------------------------------

% --------------------------------------------------
% DEFINITION
% --------------------------------------------------

% --------------------------------------------------
% DEFINITION
% --------------------------------------------------

% --------------------------------------------------
% DEFINITION
% --------------------------------------------------

% --------------------------------------------------
% DEFINITION
% --------------------------------------------------

% --------------------------------------------------
% DEFINITION
% --------------------------------------------------

% --------------------------------------------------
% DEFINITION
% --------------------------------------------------

% --------------------------------------------------
% DEFINITION
% --------------------------------------------------

% --------------------------------------------------
% DEFINITION
% --------------------------------------------------

% --------------------------------------------------
% DEFINITION
% --------------------------------------------------

% --------------------------------------------------
% DEFINITION
% --------------------------------------------------

% --------------------------------------------------
% DEFINITION
% --------------------------------------------------

% --------------------------------------------------
% DYNARE
% --------------------------------------------------

\newpage

\subsection{Dynare Program}

This section presents the \texttt{mod} file used in \dynare to solve the model in section \ref{sec:model}.

\vspace*{-1cm}

{\singlespacing

\begin{verbatim} 
	
	% command to run dynare and write
	% a new file with all the choices:
	% dynare NK_Inv_MonPol savemacro=NK_Inv_MonPol_FINAL.mod
	
	% -------------------------------------------------- %
	% MODEL OPTIONS                                      %
	% -------------------------------------------------- %
	
	% Productivity Shock ON/OFF
	@#define ZA_SHOCK    = 1
	% Productivity Shock sign: +/-
	@#define ZA_POSITIVE = 1
	% Monetary Shock ON/OFF
	@#define ZM_SHOCK    = 1
	% Monetary Shock sign: +/-
	@#define ZM_POSITIVE = 1
	
	% -------------------------------------------------- %
	% ENDOGENOUS VARIABLES                               %
	% -------------------------------------------------- %
	
	var
	PIt       ${\tilde{\pi}}$     (long_name='Inflation Rate')
	Pt        ${\hat{P}}$         (long_name='Price Level')
	LAMt      ${\tilde{\lambda}}$ (long_name='Real Marginal Cost')
	Ct        ${\hat{C}}$         (long_name='Consumption')
	Lt        ${\hat{L}}$         (long_name='Labor')
	Rt        ${\hat{R}}$         (long_name='Interest Rate')
	Kt        ${\hat{K}}$         (long_name='Capital')
	It        ${\hat{I}}$         (long_name='Investment')
	Wt        ${\hat{W}}$         (long_name='Wage')
	ZAt       ${\hat{Z}^A}$       (long_name='Productivity')
	Yt        ${\hat{Y}}$         (long_name='Production')
	ZMt       ${\hat{Z}^M}$       (long_name='Monetary Policy')
	;
	
	% -------------------------------------------------- %
	% LOCAL VARIABLES                                    %
	% -------------------------------------------------- %
	
	% the steady state variables are used as local 
	variables for the linear model.
	
	model_local_variable
	
	% steady state variables used as locals:
	P
	PI
	ZA
	ZM
	R
	LAM
	W
	Y
	C
	K
	L
	I
	
	% local variables:
	RHO % Steady State Discount Rate
	;
	
	% -------------------------------------------------- %
	% EXOGENOUS VARIABLES                                %
	% -------------------------------------------------- %
	
	varexo
	epsilonA ${\varepsilon_A}$   (long_name='productivity shock')
	epsilonM ${\varepsilon_M}$   (long_name='monetary shock')
	;
	
	% -------------------------------------------------- %
	% PARAMETERS                                         %
	% -------------------------------------------------- %
	
	parameters
	SIGMA   ${\sigma}$     (long_name='Relative Risk Aversion')
	PHI     ${\phi}$       (long_name='Labor Disutility Weight')  
	VARPHI  ${\varphi}$    (long_name='Marginal Disutility of Labor Supply')
	BETA    ${\beta}$      (long_name='Intertemporal Discount Factor')
	DELTA   ${\delta}$     (long_name='Depreciation Rate')
	ALPHA   ${\alpha}$     (long_name='Output Elasticity of Capital')
	PSI     ${\psi}$       (long_name='Elasticity of 
	Substitution between Intermediate Goods')
	THETA   ${\theta}$     (long_name='Price Stickness Parameter')
	gammaR  ${\gamma_R}$   (long_name='Interest-Rate Smoothing Parameter')
	gammaPI ${\gamma_\pi}$ (long_name='Interest-Rate 
	Sensitivity to Inflation')
	gammaY  ${\gamma_Y}$   (long_name='Interest-Rate Sensitivity to Product')
	% maybe it's a local var, right? RHO ${\rho}$ 
	(long_name='Steady State Discount Rate')
	rhoA    ${\rho_A}$     (long_name='Autoregressive 
	Parameter of Productivity Shock')
	rhoM    ${\rho_M}$     (long_name='Autoregressive 
	Parameter of Monetary Policy Shock')
	thetaC  ${\theta_C}$   (long_name='Consumption weight 
	in Output')
	thetaI  ${\theta_I}$   (long_name='Investment weight 
	in Output')
	
	% -------------------------------------------------- % 
	% standard errors of stochastic shocks               %
	% -------------------------------------------------- %
	
	sigmaA ${\sigma_A}$   (long_name='Productivity-Shock 
	Standard Error')
	sigmaM ${\sigma_M}$   (long_name='Monetary-Shock 
	Standard Error')
	;
	
	% -------------------------------------------------- %
	% parameters values                                  %
	% -------------------------------------------------- % 
	
	SIGMA   = 2       ; % Relative Risk Aversion
	PHI     = 1       ; % Labor Disutility Weight
	VARPHI  = 1.5     ; % Marginal Disutility of Labor 
	Supply
	BETA    = 0.985   ; % Intertemporal Discount Factor
	DELTA   = 0.025   ; % Depreciation Rate
	ALPHA   = 0.35    ; % Output Elasticity of Capital
	PSI     = 8       ; % Elasticity of Substitution 
	between Intermediate Goods
	THETA   = 0.8     ; % Price Stickness Parameter
	gammaR  = 0.79    ; % Interest-Rate Smoothing Parameter
	gammaPI = 2.43    ; % Interest-Rate Sensitivity 
	to Inflation
	gammaY  = 0.16    ; % Interest-Rate Sensitivity to 
	Product
	% maybe it's a local var, right? RHO = 1/(1+Rs); 
	% Steady State Discount Rate
	rhoA    = 0.95    ; % Autoregressive Parameter of 
	Productivity Shock
	rhoM    = 0.9     ; % Autoregressive Parameter of 
	Monetary Policy Shock
	thetaC  = 0.8     ; % Consumption weight in Output
	thetaI  = 0.2     ; % Investment weight in Output
	
	% -------------------------------------------------- % 
	% standard errors values                             %
	% -------------------------------------------------- %
	
	sigmaA = 0.01   ; % Productivity-Shock Standard Error
	sigmaM = 0.01   ; % Monetary-Shock Standard Error
	
	% -------------------------------------------------- %
	% MODEL                                              %
	% -------------------------------------------------- %
	
	model(linear);
	
	% First, the steady state variables as local varibles, 
	for the log-linear use:
	
	#Ps   = 1 ;
	#PIs  = 1 ;
	#ZAs  = 1 ;
	#ZMs  = 1 ;
	#Rs   = Ps*(1/BETA-(1-DELTA)) ;
	#LAMs = Ps*(PSI-1)/PSI ;
	#Ws   = (1-ALPHA)*(LAMs*ZAs*(ALPHA/Rs)^ALPHA)^
	(1/(1-ALPHA)) ;
	#Ys   = ((Ws/(PHI*Ps))*((Ws/((1-ALPHA)*LAMs))^PSI)*(Rs/
	(Rs-DELTA*ALPHA*LAMs))^SIGMA)^(1/(PSI+SIGMA)) ;
	#Cs   = ((Ws/(PHI*Ps))*((1-ALPHA)*Ys*LAMs/Ws)^
	(-PSI))^(1/SIGMA) ;
	#Ks   = ALPHA*Ys*LAMs/Rs ;
	#Ls   = (1-ALPHA)*Ys*LAMs/Ws ;
	#Is   = DELTA*Ks ;
	#RHO  = 1/(1+Rs) ;
	
	% -------------------------------------------------- % 
	% MODEL EQUATIONS                                    %
	% -------------------------------------------------- %
	
	% Second, the log-linear model:
	
	% 01 %
	[name='Gross Inflation Rate']
	PIt = Pt - Pt(-1) ;
	
	% 02 %
	[name='New Keynesian Phillips Curve']
	PIt = RHO*PIt(+1)+LAMt*(1-THETA)*(1-THETA*RHO)/THETA ;
	
	% 03 %
	[name='Labor Supply']
	VARPHI*Lt + SIGMA*Ct = Wt - Pt ;
	
	% 04 %
	[name='Household Euler Equation']
	Ct(+1) - Ct = (Rt(+1)-Pt(+1))*BETA*Rs/(SIGMA*Ps) ;
	
	% 05 %
	[name='Law of Motion for Capital']
	Kt = (1-DELTA)*Kt(-1) + DELTA*It ;
	
	% 06 %
	[name='Real Marginal Cost']
	LAMt = ALPHA*Rt + (1-ALPHA)*Wt - ZAt - Pt ;
	
	% 07 %
	[name='Production Function']
	Yt = ZAt + ALPHA*Kt(-1) + (1-ALPHA)*Lt ;
	
	% 08 %
	[name='Marginal Rates of Substitution of Factors']
	Kt(-1) - Lt = Wt - Rt ;
	
	% 09 %
	[name='Market Clearing Condition']
	Yt = thetaC*Ct + thetaI*It ;
	
	% 10 %
	[name='Monetary Policy']
	Rt = gammaR*Rt(-1) + (1 - gammaR)*(gammaPI*PIt + 
	gammaY*Yt) + ZMt ;
	
	% 11 %
	[name='Productivity Shock']
	@#if ZA_POSITIVE == 1
	ZAt = rhoA*ZAt(-1) + epsilonA ;
	@#else
	ZAt = rhoA*ZAt(-1) - epsilonA ;
	@#endif
	
	% 12 %
	[name='Monetary Shock']
	@#if ZM_POSITIVE == 1
	ZMt = rhoM*ZMt(-1) + epsilonM ;
	@#else
	ZMt = rhoM*ZMt(-1) - epsilonM ;
	@#endif
	
	end;
	
	% -------------------------------------------------- % 
	% STEADY STATE                                       %
	% -------------------------------------------------- % 
	
	steady_state_model ;
	
	% in the log-linear model, all steady state variables
	are zero (the variation is zero):
	
	PIt  = 0 ;
	Pt   = 0 ;
	LAMt = 0 ;
	Ct   = 0 ;
	Lt   = 0 ;
	Rt   = 0 ;
	Kt   = 0 ;
	It   = 0 ;
	Wt   = 0 ;
	ZAt  = 0 ;
	Yt   = 0 ;
	ZMt  = 0 ;
	
	end;
	
	% compute the steady state
	steady;
	check(qz_zero_threshold=1e-20);
	
	% -------------------------------------------------- % 
	% SHOCKS                                             %
	% -------------------------------------------------- % 
	
	shocks; 
	
	% Productivity Shock
	@#if ZA_SHOCK == 1
	var    epsilonA;
	stderr sigmaA;
	@#endif
	
	% Monetary Shock
	@#if ZM_SHOCK == 1
	var    epsilonM;
	stderr sigmaM;
	@#endif
	
	end;
	
	stoch_simul(irf=80, order=1, qz_zero_threshold=1e-20) 
	ZAt ZMt Yt Pt PIt LAMt Ct Lt Rt Kt It Wt  ;
	
	% -------------------------------------------------- % 
	% LATEX OUTPUT                                       %
	% -------------------------------------------------- % 
	
	write_latex_definitions;
	write_latex_parameter_table;
	write_latex_original_model;
	write_latex_dynamic_model;
	write_latex_static_model;
	write_latex_steady_state_model;
	collect_latex_files;
	
\end{verbatim}

} % end \singlespacing

% \newpage

\begin{comment}

% --------------------------------------------------
% LATEX
% --------------------------------------------------

\subsection{\LaTeX}

\subsubsection*{Commands}

\begin{itemize}
	
	\item cancel line in equation: \com{cancel}
	
	%	\cancel   draws a diagonal line (slash) through its argument.
	%	\bcancel  uses the negative slope (a backslash).
	%	\xcancel  draws an X (actually \cancel plus \bcancel).
	%	\cancelto{〈value〉}{〈expression〉} draws a diagonal arrow through the 〈expression〉, pointing to the 〈value〉.
	
	\item space before align: \com{vspace\{-1cm\}} % \vspace*{-1cm}
	
	\item correct paragraph overfull: \com{sloppy}
	
	\item indices: $i,j,k,\ell$
	
	\item hats: \( \overline{abc}, \widetilde{abc}, \widehat{abc}, \overrightarrow{abc}, \overleftarrow{abc}, \sqrt[n]{abc}, \xrightarrow{abc}, \xrightarrow{some text}\)
	
	\item accents: \(\acute{a}, \check{a}, \grave{a}, \widetilde{a}, \hat{a}, \breve{a}, \overline{a}, \bar{a}, \vec{a}, \dot{a}, \ddot{a}, \mathring{a}, \imath, \jmath\)
	
	\item symbols:
	
	checkmark: \checkmark 
	
	dagger: $\dagger$
	
	definition symbol: $\coloneq$
	
	\item index before the variable:
	\begin{align*}
		& + \prescript{NR}{}{C}^{\alpha}_{t+1} + \prescript{}{NR}{C}^{\alpha}_{t+1} + \prescript{}{\mathcal{nr}}{C}^{\alpha}_{t+1}        
		\\
		& + NRC^{\alpha}_{t+1} + \mathcal{nr}C^{\alpha}_{t+1} + nrC^{\alpha}_{t+1}
		\\
		& + \textsc{nrc}^{\alpha}_{t+1} + \mathscr{NR}C^{\alpha}_{t+1} + \mathcal{nr}C^{\alpha}_{t+1}
		\\
		& + \prescript{}{\mathscr{NR}}{C}^{\alpha}_{t+1} + \prescript{\mathscr{NR}}{}{C}^{\alpha}_{t+1} + {C}^{\mathscr{NR},\alpha}_{t+1} 
		\\
		& + {C}^{\textsc{nr},\alpha}_{t+1} + {C}^{\alpha}_{\textsc{nr},t+1} + \textsc{nr}C^{\alpha}_{t+1}
	\end{align*}
	
	\item summation and product operator:
	
	\begin{align*}
		\sum_{s=0}^{\infty} \frac{\theta^s}{\prod_{k=0}^{s-1} (1+R_{t+k})}
	\end{align*}
	
	\begin{align*}
		\text{Term for } s = 0: \frac{\theta^0}{\prod_{k=0}^{-1} (1+R_{t+k})} = \theta^0 = 1
	\end{align*}
	
	\begin{align*}
		\text{Term for } s = 1: \frac{\theta^1}{\prod_{k=0}^{0} (1+R_{t+k})} = \frac{\theta^1}{1+R_{t+0}} = \frac{\theta}{1+R_t}
	\end{align*}
	
\end{itemize}

\newpage

\subsubsection*{Font Styles in Math Mode}

\begin{itemize}
	
	\item San Serif Style: \com{mathsf}
	\begin{gather*}
		\mathsf{
			A B C D E F G H I J K L M N O P Q R S T U V W X Y Z
		}\\
		\mathsf{
			a b c d e f g h i j k l m n o p q r s t u v w x y z
		}\\
		\mathsf{
			1 2 3 4 5 6 7 8 9 0
		}
	\end{gather*}
	
	\item Fraktur Style: \com{mathfrak}
	\begin{gather*}
		\mathfrak{
			A B C D E F G H I J K L M N O P Q R S T U V W X Y Z
		}\\
		\mathfrak{
			a b c d e f g h i j k l m n o p q r s t u v w x y z
		}\\
		\mathfrak{
			1 2 3 4 5 6 7 8 9 0
		}
	\end{gather*}
	
	\item Fraktur-bold Style: \com{mathbffrak}
	\begin{gather*}
		\mathbffrak{
			A B C D E F G H I J K L M N O P Q R S T U V W X Y Z
		}\\
		\mathbffrak{
			a b c d e f g h i j k l m n o p q r s t u v w x y z
		}\\
		\mathbffrak{
			1 2 3 4 5 6 7 8 9 0
		}
	\end{gather*}
	
	\item Calligraphic Style: \com{mathcal}
	\begin{gather*}
		\mathcal{
			A B C D E F G H I J K L M N O P Q R S T U V W X Y Z
		}\\
		\mathcal{
			a b c d e f g h i j k l m n o p q r s t u v w x y z
		}
	\end{gather*}
	
	\item Calligraphic-bold Style: \com{mathbfcal}
	\begin{gather*}
		\mathbfcal{
			A B C D E F G H I J K L M N O P Q R S T U V W X Y Z
		}\\
		\mathbfcal{
			a b c d e f g h i j k l m n o p q r s t u v w x y z
		}
	\end{gather*}
	
	\item Script Style: \com{mathscr}
	\begin{gather*}
		\mathscr{
			A B C D E F G H I J K L M N O P Q R S T U V W X Y Z
		}
	\end{gather*}
	
	\item Script-bold Style: \com{mathbfscr}
	\begin{gather*}
		\mathbfscr{
			A B C D E F G H I J K L M N O P Q R S T U V W X Y Z
		}
	\end{gather*}
	
	\item Blackboard-bold Style: \com{mathbb}
	\begin{gather*}
		\mathbb{
			A B C D E F G H I J K L M N O P Q R S T U V W X Y Z
		}\\
		\mathbb{
			a b c d e f g h i j k l m n o p q r s t u v w x y z
		}\\
		\mathbb{
			1 2 3 4 5 6 7 8 9 0
		}
	\end{gather*}
	
\end{itemize}

\newpage

\subsubsection*{Greek Letters}

\vspace*{0.5cm}

\begin{center}
	
\begin{longtblr}[
	label = {table:all-greek-letters},
	caption = {Greek Letters},
	remark{Source} = {The Author.},
	]{rowhead = 1,
	colspec = {l|l|l}}
		\hline[2pt]
		\textbf{Lower Case}                          & \textbf{Upper Case}                          & \textbf{Variation}                                    \\
		\hline[2pt]
		$\alpha,\bm{\alpha}$ $\backslash$alpha       & $A,\bm{A}$                                   &                                                       \\
		$\beta,\bm{\beta}$ $\backslash$beta          & $B,\bm{B}$                                   &                                                       \\
		$\gamma,\bm{\gamma}$ $\backslash$gamma       & $\Gamma,\bm{\Gamma}$ $\backslash$Gamma       &                                                       \\
		$\delta,\bm{\delta}$ $\backslash$delta       & $\Delta,\bm{\Delta}$ $\backslash$Delta       &                                                       \\
		$\epsilon,\bm{\epsilon}$ $\backslash$epsilon & $E,\bm{E}$                                   & $\varepsilon,\bm{\varepsilon}$ $\backslash$varepsilon \\
		$\zeta,\bm{\zeta}$ $\backslash$zeta          & $Z,\bm{Z}$                                   &                                                       \\
		$\eta,\bm{\eta}$ $\backslash$eta             & $H,\bm{H}$                                   &                                                       \\
		$\theta,\bm{\theta}$ $\backslash$theta       & $\Theta,\bm{\Theta}$ $\backslash$Theta       & $\vartheta,\bm{\vartheta}$ $\backslash$vartheta       \\
		$\iota,\bm{\iota}$ $\backslash$iota          & $I,\bm{I}$                                   &                                                       \\
		$\kappa,\bm{\kappa}$ $\backslash$kappa       & $K,\bm{K}$                                   & $\varkappa,\bm{\varkappa}$ $\backslash$varkappa       \\
		$\lambda,\bm{\lambda}$ $\backslash$lambda    & $\Lambda,\bm{\Lambda}$ $\backslash$Lambda    &                                                       \\
		$\mu,\bm{\mu}$ $\backslash$mu                & $M,\bm{M}$                                   &                                                       \\
		$\nu,\bm{\nu}$ $\backslash$nu                & $N,\bm{N}$                                   &                                                       \\
		$\xi,\bm{\xi}$ \com{xi} & $\Xi,\bm{\Xi}$ \com{Xi} & \\
		$o,\bm{o}$ (omicron)                         & $O,\bm{O}$                                   &                                                       \\
		$\pi,\bm{\pi}$ $\backslash$pi                & $\Pi,\bm{\Pi}$ $\backslash$Pi                & $\varpi,\bm{\varpi}$ $\backslash$varpi                \\
		$\rho,\bm{\rho}$ $\backslash$rho             & $P,\bm{P}$                                   & $\varrho,\bm{\varrho}$ $\backslash$varrho             \\
		$\sigma,\bm{\sigma}$ $\backslash$sigma       & $\Sigma,\bm{\Sigma}$ $\backslash$Sigma       & $\varsigma,\bm{\varsigma}$ $\backslash$varsigma       \\
		$\tau,\bm{\tau}$ $\backslash$tau             & $T,\bm{T}$                                   &                                                       \\
		$\upsilon,\bm{\upsilon}$ $\backslash$upsilon & $\Upsilon,\bm{\Upsilon}$ $\backslash$Upsilon &                                                       \\
		$\phi,\bm{\phi}$ $\backslash$phi             & $\Phi,\bm{\Phi}$ $\backslash$Phi             & $\varphi,\bm{\varphi}$ $\backslash$varphi             \\
		$\chi,\bm{\chi}$ $\backslash$chi             & $X,\bm{X}$                                   &                                                       \\
		$\psi,\bm{\psi}$ $\backslash$psi             & $\Psi,\bm{\Psi}$ $\backslash$Psi             &                                                       \\
		$\omega,\bm{\omega}$ $\backslash$omega       & $\Omega,\bm{\Omega}$ $\backslash$Omega       &                                                       \\
		\hline[2pt]
	\end{longtblr}
	
\end{center}

\newpage

% --------------------------------------------------
% TODO LIST
% --------------------------------------------------

\subsection{ToDo List}

% \todototoc
\listoftodos


% --------------------------------------------------
% EPIGRAPHS
% --------------------------------------------------

\subsection{Epigraphs}

\begin{flushright}
	\textit{To be yourself in a world that is constantly trying to \\ make you something else is the greatest accomplishment. \\ --- Ralph Waldo Emerson}
\end{flushright}

\begin{flushright}
	\textit{
		The reason anyone would do this, if they could, which they can't, \\ 
		would be because they could, which they can't. \\
		--- Pickle Rick}
\end{flushright}

\begin{flushright}
	\textsf{
		\textbf{
			THE CIRCLE IS NOW COMPLETE.\\
			\textit{
				--- Darth Vader
			}		
		}
	}
\end{flushright}

% --------------------------------------------------
% DRAFTS
% --------------------------------------------------

\subsection{Drafts}

%If $\beta=\delta=1$ and $\mathbb{E}_t r_{t+1} = \mathbb{E}_t \left(\frac{R{t+1}}{P{t+1}}\right)$ (the real rate of capital return), this equation shows that the marginal rate of substitution of current consumption for future consumption equals the real rate of capital return:
%\begin{align}
%	\mathbb{E}_t \left( \frac{C_{t+1}}{C_t} \right)^\sigma & = \mathbb{E}_t \left( r_{t+1} \right) \label{eq:household present-future consumption MRS}
%\end{align}
%
%\noindent where $\pi_t$ is the Gross Inflation Rate:
%\begin{align}
%	\pi_t = \frac{P_t}{P_{t-1}} \label{eq:Gross Inflation Rate}
%\end{align}

%%%%%%%%%%%%%%%%%%%%%%%%%%%%%%%%%%%%%%%%%%%%%%%%%%
%%%%%%%%%%%%%%%%%%%%%%%%%%%%%%%%%%%%%%%%%%%%%%%%%%
%%%%%%%%%%%%%%%%%%%%%%%%%%%%%%%%%%%%%%%%%%%%%%%%%%

%Note that all firms have the same productivity, markup and marginal costs, so that the optimal price level, capital and labor demand are also the same for each firm. This way, the ($_j$) index can be dropped from optimal price, capital and labor demand equations:
%\begin{align}
%	P_{jt}^\ast & = P_t^\ast = \left( \frac{\psi}{\psi-1} \right) \mathbb{E}_t \sum_{i=0}^{\infty} (\beta \theta)^i \Lambda_{t+i}
%	\label{eq:IGF Optimal Price Level}                                                                                    \\
%	K_{jt}   & = K_t = \alpha \Lambda_t \frac{Y_t}{R_t}
%	\label{eq:capital demand MC}                                                                                          \\
%	L_{jt}   & = L_t = (1-\alpha) \Lambda_t \frac{Y_t}{W_t}
%	\label{eq:labor demand MC}
%\end{align}

%%%%%%%%%%%%%%%%%%%%%%%%%%%%%%%%%%%%%%%%%%%%%%%%%%
%%%%%%%%%%%%%%%%%%%%%%%%%%%%%%%%%%%%%%%%%%%%%%%%%%
%%%%%%%%%%%%%%%%%%%%%%%%%%%%%%%%%%%%%%%%%%%%%%%%%%

%Considering the household and the firm problems, there is a set of endogenous variables including consumption $C_t$, labor $L_t$, capital $K_t$, real rental rate of capital $r$, real wage $w$, and output $Y$, which is determined by a set of exogenous variables including technology $A$ and initial capital $K$. Competitive equilibrium occurs when supply equals demand in all three markets: goods, labor, and capital. The equilibrium values of the endogenous variables are determined by the prices of the three markets: the real rental rate of capital, the real wage rate, and the price of the consumption good (normalized to one).
%
%From the household problem, we have:
%\begin{align}
%	U_l &= u_C(C^\ast,l^\ast) w^\ast     \label{eq:hp1} \\
%	C^\ast &= w^\ast (h-l^\ast) + r^\ast K  \label{eq:hp2} 
%\end{align}
%
%From the firm problem, we have:
%\begin{align}
%	r^\ast &= A F_K(K_F^\ast,L_F^\ast) \label{eq:fp1} \\ 
%	w^\ast &= A F_L(K_F^\ast,L_F^\ast) \label{eq:fp2} 
%\end{align}
%
%And the market clearing conditions are:
%\begin{align}
%	K_F^\ast &= K      \label{eq:mcc1} \\ 
%	L_F^\ast &= h-l^\ast  \label{eq:mcc2}
%\end{align}
%
%We now substitute \ref{eq:mcc1} and \ref{eq:mcc2} in \ref{eq:fp1} and \ref{eq:fp2} to get:
%\begin{align}
%	r^\ast &= A F_K(K,h-l^\ast) \label{eq:fp1b} \\
%	w^\ast &= A F_L(K,h-l^\ast) \label{eq:fp2b}
%\end{align}
%
%Combining \ref{eq:hp1} and \ref{eq:fp2b}, we have:
%\begin{align}
%	U_l(C^\ast,l^\ast) &= A F_L(K,h-l^\ast) u_C(C^\ast,l^\ast) w^\ast
%\end{align}
%
%Optimal profits must equal zero, then:
%\begin{align}
%	A F(K,h-l^\ast) = r^\astK + w^\ast(h-l^\ast) \label{eq:opz}
%\end{align}
%
%We can plug \ref{eq:opz} in \ref{eq:hp1} to get:
%\begin{align}
%	C^\ast &= A F(K,h-l^\ast)
%\end{align}
%
%Hence, the CE can be uniquely characterized by:
%\begin{align}
%	U_l(C^\ast,l^\ast) &= A F_L(K,h-l^\ast) u_C(C^\ast,l^\ast) w^\ast  \label{eq:ce1} \\
%	C^\ast &= A F(K,h-l^\ast)                     \label{eq:ce2} 
%\end{align}
%
%Equations \ref{eq:ce1} and \ref{eq:ce2} form a system of two equations and two variables $\{C^\ast,l^\ast\}$. Given the parameters and functional forms, it can be solved and all other variables determined with the previous equations.

%%%%%%%%%%%%%%%%%%%%%%%%%%%%%%%%%%%%%%%%%%%%%%%%%%
%%%%%%%%%%%%%%%%%%%%%%%%%%%%%%%%%%%%%%%%%%%%%%%%%%
%%%%%%%%%%%%%%%%%%%%%%%%%%%%%%%%%%%%%%%%%%%%%%%%%%


%%%%%%%%%%%%%%%%%%%%%%%%%%%%%%%%%%%%%%%%%%%%%%%%%%
%%%%%%%%%%%%%%%%%%%%%%%%%%%%%%%%%%%%%%%%%%%%%%%%%%
%%%%%%%%%%%%%%%%%%%%%%%%%%%%%%%%%%%%%%%%%%%%%%%%%%


%%%%%%%%%%%%%%%%%%%%%%%%%%%%%%%%%%%%%%%%%%%%%%%%%%
%%%%%%%%%%%%%%%%%%%%%%%%%%%%%%%%%%%%%%%%%%%%%%%%%%
%%%%%%%%%%%%%%%%%%%%%%%%%%%%%%%%%%%%%%%%%%%%%%%%%%
	
\end{comment}	
	
\end{document}