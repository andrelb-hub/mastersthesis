% --------------------------------------------------
% DOCUMENT CLASS
% --------------------------------------------------

\documentclass[../thesis.tex]{subfiles}

\begin{document}

	\newpage
	
	\section{Final Remarks}\label{sec:final-remarks}

The primary objective of this thesis was to construct a DSGE model for evaluating the influence of monetary policy on regions within the same country that possess distinct productive structures. The impulse response functions revealed discernible differences in the intensity of reactions, attributable to variations in capital elasticity and technological levels among firms in different regions, while the direction of the reactions remained consistent.

From this information, one can infer that a national monetary policy may not uniformly exhibit effectiveness across all regions and may not be universally optimal and will have different impacts on different regions. For one region, the monetary policy decision may be just what it needed, while to the other it may be too harsh or too soft. Regardless of the decision, it is evident that a country with diverse regions should consider that the interest rate, among many available monetary policy mechanisms, is not universally optimal and will have different impacts on different regions.

Consequently, contemplating the implementation of alternative policies in parallel may be necessary to ensure the desired effects are achieved. Alternatively, the formulation of monetary policy should consider regional variables, allowing the intensity of the policy to be weighted according to the characteristics of each region.

Considering the potential benefits of coordinated fiscal and monetary policies, it becomes imperative to explore avenues for collaboration between fiscal and monetary authorities. Given the existence of regional fiscal authorities, one plausible approach is to introduce regional fiscal policies that complement monetary policy measures. In this context, a historical example comes to mind: the issuance of State Bonds by Brazilian regional governments. Such regional mechanisms could be strategically leveraged in coordination with monetary policies to achieve the desired economic effects. But this is another story, for another thesis.

While the model presented here lays the groundwork for understanding the dynamics of regional economies, future studies could further enrich our understanding by incorporating additional elements to this framework, such as: % The model presented here incorporates features characteristic of a New Keynesian model. For future studies, it should be considered to include additional elements, such as:
\begin{enumerate*}[label=(\arabic*)]
	\item non-Ricardian households, given that a significant portion of the Brazilian population lacks access to credit; 
	\item habit formation, as this feature provides a more accurate description of household behavior;
	\item labor market, as rigidities within it contribute to a better alignment of the model with reality;
	\item a fiscal authority, recognizing the significance of government decisions on private agents;
	\item adjustment costs on investment, acknowledging that higher investments can increase its overall expense;
	\item inclusion of bonds and other assets, as there are various financial products within the economy;
	\item consideration of the foreign market, recognizing the influence of other global economies on internal decisions.
\end{enumerate*} \hfill $\blacksquare$

:)

\begin{comment}
	The model presented here contemplates some features that characterize a New Keynesian model. For future studies, it should be considered to implement more elements, such as 
	\begin{enumerate*}[label=(\arabic*)]
		\item non-Ricardiand households, as a considerable part of Brasilian population has no access to credit;
		\item habit formation, as this feature describes better the household behavior;
		\item labor market, as its rigidities contribute for a better adherence of the model to reality;
		\item a fiscal authority, as the influence of government decisions are important to the private agents;
		\item adjustment costs on investment, considering that the higher the investment, the more expensive it becomes;
		\item bonds and other assets, as there are different financial products to choose from in the economy;
		\item foreign market, as the other economies in the world can influence the internal decisions.
	\end{enumerate*}
\end{comment}

% \subsection{Future Challenges}



 

\end{document}