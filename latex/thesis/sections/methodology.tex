% --------------------------------------------------
% DOCUMENT CLASS
% --------------------------------------------------

\documentclass[../thesis.tex]{subfiles}

\begin{document}

\newpage

\section{Methodology}\label{sec:methodology}

The DSGE methodology, as the name implies, involves the utilization of a Dynamic and Stochastic General Equilibrium model. This model outlines the problem to be addressed, requiring the definition of agents, variables, and parameters. In this research, the objective is to assess the impact of monetary policy on regional gross domestic product using the Canonical New Keynesian structure, as proposed by \textcite{solis-garcia_ucb_2022}. The structure comprises four representative agents: a household, a retail firm, a continuum of wholesale firms, and a monetary authority. It incorporates key elements of the New Keynesian theory, including monopolistic competition among wholesale firms, the price stickiness they encounter, and the consequential role of monetary policy in the short run. % The role of each agent, variable and parameter will be discussed in detail in the next section.

	% The DSGE model has several steps, and therefore, the methodology section needs a table of contents of its own.

\begin{comment}
	
% local table os contents:
{
	\setlength{\parskip}{1pt}
	\singlespacing
	
	\etocsettocdepth{3}
	\localtableofcontents
}

\end{comment}

\end{document}