% --------------------------------------------------
% DOCUMENT CLASS
% --------------------------------------------------

\documentclass[../thesis.tex]{subfiles}

\begin{document}

\newpage

\section{Literature Review}\label{sec:literature-review}

This section provides a literature review, exploring the intersection between Regional Economics and Macroeconomics, emphasizing the importance of monetary policy, and delving into the applicability of DSGE models to address diverse economic challenges, including regional and monetary dilemmas. The discussion also underscores the need for a clear definition and methodological framework in utilizing DSGE models.

% \subsection*{Macroeconomics and Regional Economics}

The assessment by \textcite{rickman_modern_2010} on the importance of the link between Macroeconomics and Regional Economics was made at a time when the use of DSGE models to investigate regional issues was not yet common. Since then, several studies have addressed this connection.

Initially, we present two works that served as inspiration for the present research. The first, developed by \textcite{costa_junior_dsge_2022}, investigates the impacts of fiscal policy on the state of Goiás, considering the other states of the nation. In this work, the authors develop a regionalized and open structure, individualizing a Brazilian state from the rest, considering both a national and a state fiscal authority; state expenses and revenues are disaggregated, and thus, the authors seek to identify whether there are differences between the impacts of a tax exemption in the state under study compared to the others. With the model calibrated to data from 2003 to 2019, the authors demonstrate that there is indeed a difference in state performance due to the distinction of the tax exemption occurring in the state or in the rest of the country.

The second work also presents a DSGE model, but with the objective of evaluating whether there are differences in the effects of Foreign Direct Investment (FDI), considering its location. The model developed by \textcite{mora_fdi_2019} encompasses an open economy with the main region (Bogotá, 25\% of the national GDP) and the rest of the country (Colombia), two types of households\footnote{ Ricardian and non-Ricardian agents.}, habit formation, capital adjustment costs, as well as typical elements of a New Keynesian (NK) model\footnote{ nominal price rigidity, monopolistic competition, non-neutrality of monetary policy in the short term.}. With the model calibrated to data from 2002 to 2015, the authors demonstrate that there is indeed a difference in the effects of FDI depending on the region where it is applied, such that when applied in the rest of the country, there are growth effects that spread throughout the country through spillovers, including to the main region.

Both works aim to, despite dealing with distinct causes (fiscal policy and FDI), verify whether differences exist when the cause occurs in one of the two different modeled regions. Additionally, they share the same modeling approach, that of a Dynamic and Stochastic General Equilibrium (DSGE). And this was the advancement that \textcite{rickman_modern_2010} wanted to see happen: the use of DSGE models to address regional questions.

% \subsection*{Monetary Policy}

DSGE models are widely employed within the macroeconomic literature to examine the effects of monetary policy on macroeconomic aggregates, as pointed by \textcite{gali_monetary_2015}. In this context, it is important to add to the review the papers that develop models describing the monetary policy.

\textcite{smets_estimated_2003} and \textcite{smets_shocks_2007} present models that evaluate various types of shocks in the Eurozone and the United States, respectively. \textcite{walque_financial_2010} assess the role of the banking sector in market liquidity recovery, considering the endogenous possibilities of agent default.

\textcite{vinhado_politica_2016} employ a model with financial frictions to examine the transmission of monetary policy to the banking sector and economic activity. The results demonstrate that the banking sector plays a significant role in economic activity and impacts the outcomes of monetary policy by having to adjust the bank spread in response to changes in the interest rate or reserve requirements.

\textcite{soltani_investigating_2021} investigate financial and monetary shocks on macroeconomic variables, with special attention to the role of banks. For this analysis, the model considers an economy with a banking sector. The results indicate that banking activity can influence the effects of economic policies.

\textcite{holm_transmission_2021} study the transmission of monetary policy to household consumption, estimating the response of consumption, income, and savings. They utilize a heterogeneous agent New Keynesian model (HANK). The results demonstrate that a restrictive monetary policy prompts households with lower liquidity to reduce consumption as disposable income starts to decline, while households with average liquidity save less or borrow more. The study also highlights the differences in consumption changes between savers and borrowers in the face of a monetary policy alteration.

\textcite{capeleti_countercyclical_2022} evaluate the effects of pro-cyclical and counter-cyclical credit expansions by public banks on economic growth. The model implements a banking sector with public and private banks competing in a Cournot oligopoly. The results show that the supply of public credit has a stronger effect when the policy is counter-cyclical.

% \subsection*{Macroeconomic Modeling}

The literature on DSGE modeling is extensive, as this methodology allows the formulation of various questions and their answers through a general equilibrium model. This includes the aforementioned topics and, also, labor market, as explored by \textcite{ribeiro_alongamento_2023}; the real estate market, as studied by \textcite{albuquerquemello_mercado_2018}; and even deforestation, as investigated by \textcite{pereira_desmatamento_2013}. As remarked by \textcite{solis-garcia_ucb_2022}: \textit{if you have a cohesive economic idea, you can put it in terms of a DSGE model}. %\footnote{ \textit{"If you have a cohesive economic idea, you can put it in terms of a DSGE model."}}

The works of \textcite{costa_junior_understanding_2016}, \textcite{solis-garcia_ucb_2022}, \textcite{bergholt_basic_2012}, and \textcite{gali_monetary_2015}, between others, are essential materials for macroeconomic modeling theory, as they guide the reader in developing a DSGE model step-by-step. \textcite{costa_junior_understanding_2016} starts from a Real Business Cycles (RBC) model and chapter by chapter adds elements of New Keynesian (NK) theory to the model. \textcite{solis-garcia_ucb_2022} focuses on the mathematical details necessary to develop a DSGE model, beginning with a RBC model and turning it into a canonical NK model. \textcite{bergholt_basic_2012} discusses the key elements of a New Keynesian model and also demonstrates the necessary programming to run the model using the \dynare{} software. \textcite{gali_monetary_2015} shows the evolution from an RBC model to an NK model, adding complexity with each chapter.

% \footnote{ more details about the \dynare{} software in section \ref{sec:dynare}. }

% \subsection*{Macroeconomic Modeling with Regions}

% In this work, the focus is on the utilization of structural modeling with regions to investigate the existence of a relationship between a macroeconomic variable and a regional one.

Among the works employing DSGE modeling with regions, beside the already mentioned before, there is the study by \textcite{tamegawa_two-region_2012}, which assesses the effects of fiscal policy on two regions using a model featuring two types of households, firms, banks, a national government, and a regional government. Using literature parameters to calibrate the model, the results indicate that indeed there are differences in the effects of fiscal policy depending on which region implements it. It is important to note that the difference between a macroeconomic model and a regional one lies in the fact that in the former, aggregate variables are considered only at the national level, whereas in the latter, both national and regional variables are considered, and depending on the size of the region, the latter might not be able to affect the former, as explained by \textcite{tamegawa_constructing_2013}. 	A framework to assess the economic evolution of a region in Japan is constructed by \textcite{okano_development_2015}, with the aim of identifying the causes of stagnation in the Kansai region.

In a similar vein of demonstrating regional relationships, \textcite{pytlarczyk_estimated_2005} investigates aspects of the European Monetary Union (EMU), focusing on the German economy, using a structural model with two regions; \textcite{gali_optimal_2005} also evaluates the functioning of the EMU, but with a model where regions form a unitary continuum, such that one region cannot affect the entire economy. \textcite{alpanda_international_2014} utilize a two-region model to assess the effects of US financial shocks on the euro area economy.

The article by \textcite{croitorov_financial_2020} seeks to identify spillovers between regions, building a model with three regions: the Euro area, the US, and the rest of the world. Similarly investigating spillovers, \textcite{corbo_maja_2020} present a regional model encompassing Sweden and the rest of the world.\footnote{ Spillovers: effects that are transmitted from one region to another due to an exogenous factor, such as being neighboring regions.}

More recently, a landmark was established by \textcite{osterno_uma_2022} in the field of regional models for the Brazilian economy: their endeavor adapted the aggregated Brazilian DSGE model developed by \textcite{castro_samba_2015} to include regional disaggregation, enabling the observation of local variable reactions in response to fiscal and monetary shocks. Using a top-down approach, the disaggregation of the main variables allows for the incorporation of regional data. The respective impulse response functions demonstrate that different regions exhibit different reactions to fiscal and monetary shocks.

% The papers mentioned here demonstrate the importance of the relationship between Regional Economics and Macroeconomics, and how both areas can benefit from the usage of DSGE models to illustrate the existing relations among regional and national variables. With these concepts in mind, next section presents the development of a DSGE model using a bottom-up approach: the model is created with built-in regions, so that its development contemplates the regional relationships in its core.

The papers mentioned here demonstrate the importance of the relationship between Regional Economics and Macroeconomics, and how both areas can benefit from the usage of DSGE models to illustrate the existing relations among regional and national variables. With these concepts in mind, the next section presents the development of a DSGE model using a bottom-up approach: the model is created with built-in regions, ensuring that regional relationships are integral to its development.

% \subsection*{DSGE Methodology}

% The DSGE methodology consists, as the name indicates, the utilization of a Dynamic and Stochastic General Equilibrium model. The model describes the problem to be solved and for that, one must defines which agents, variables and parameters will be used. In this research, the objective is to verify the impact of the monetary policy on the regional gross domestic product. For that, we will use the Canonical New Keynesian structure \textcite{solis-garcia_ucb_2022}, which consists of four representative agents: a household, a retail firm, a continnum of wholesale firms and a monetary authority; the elements that represents the New Keynesian theory: the monopolist competition between the wholesale firms, the price stickness they face, and the consequent role of monetary policy in the short run.

\begin{comment}

\hrulefill

\newpage

\textcolor{gray}{\textbf{Define the Scope}: Begin by defining the scope of your literature review. Clearly state the research questions or objectives you aim to address and the specific areas of macroeconomic modeling you'll cover.}: \textcolor{violet}{ok}

\textcolor{gray}{\textbf{Search for Relevant Literature}: Conduct a comprehensive search of academic databases, journals, books, and other reputable sources. Use relevant keywords and filters to find literature related to your research topic.}: \textcolor{violet}{ok}

\textcolor{gray}{\textbf{Organize the Material}: Categorize the literature into themes, topics, or key concepts. This organization will help you structure your review logically.}: \textcolor{violet}{ok}

\textcolor{gray}{\textbf{Summarize Key Concepts}: For each theme or concept, provide a concise summary of the theories, models, and empirical findings. Explain the main ideas and the contributions of each study.}: \textcolor{violet}{ok}

\textcolor{gray}{\textbf{Identify Gaps and Controversies}: As you review the literature, pay attention to gaps in the research. Are there areas where the literature is lacking or conflicting? Highlight these gaps and controversies.}: \textcolor{violet}{ok}

\textcolor{gray}{\textbf{Include Empirical Studies}: Discuss empirical studies related to your research topic. Summarize the methods, data sources, and key findings of these studies. Explain how empirical research has contributed to the field.}: \textcolor{violet}{ok}

\textcolor{gray}{\textbf{Relate to Your Research}: Throughout the literature review, connect the reviewed material to your own research. Explain how the existing literature informs your research questions and hypotheses.}: \textcolor{violet}{ok}

\textcolor{gray}{\textbf{Discuss Theoretical Foundations}: Describe the fundamental macroeconomic models, theories, and frameworks relevant to your research. Explain how these theories have evolved over time and influenced the field.}

\textcolor{gray}{\textbf{Emphasize Methodology}: If your research involves specific methodologies or econometric techniques, discuss them in this section. Explain the relevance of these methods within the context of macroeconomic modeling.}

\textcolor{gray}{\textbf{Cite Properly}: Ensure that you cite all the sources accurately and consistently according to the required citation style (e.g., APA, MLA, Chicago).}

\textcolor{gray}{\textbf{Use Subheadings}: Use subheadings to clearly delineate different themes or topics within the literature review. This makes it easier for your readers to follow your discussion.}

\textcolor{gray}{\textbf{Provide Critical Analysis}: Don't just summarize the literature; critically evaluate the strengths and weaknesses of the studies you've reviewed. Discuss any methodological limitations or potential biases in the research.}

\textcolor{gray}{\textbf{Keep it Relevant}: Stick to the literature that is directly relevant to your research questions. Avoid going off-topic or including excessive detail.}

\textcolor{gray}{\textbf{Conclude the Literature Review}: Summarize the key takeaways from your review. Restate the gaps in the literature and how your research will address them.}

\textcolor{gray}{\textbf{Be Concise and Clear}: Write in a clear, concise, and organized manner. Avoid overly technical jargon that might confuse readers.}

\textcolor{gray}{\textbf{Revise and Proofread}: Review and edit your literature review section for clarity, coherence, and grammatical accuracy.}

\textcolor{gray}{\textbf{Seek Feedback}: Share your literature review with your thesis advisor or a peer for feedback and suggestions for improvement.}

\textcolor{gray}{\textbf{Update as Necessary}: If you make significant changes to your research, revisit and update your literature review to reflect any new developments in the field.}

% \subsection*{Regional Economics}

% \subsection*{Regional DSGE Models}

% \subsection*{Impacts of Monetary Policy on Regional Gross Domestic Product}

\end{comment}

\end{document}