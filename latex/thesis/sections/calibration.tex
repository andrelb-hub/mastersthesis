% --------------------------------------------------
% DOCUMENT CLASS
% --------------------------------------------------

\documentclass[../thesis.tex]{subfiles}

\begin{document}

\newpage

% --------------------------------------------------
% PARAMETER CALIBRATION
% --------------------------------------------------

\subsection{Calibration}

\begin{comment}
	
Parameters are calibrated in order to reproduce Brazilian characteristics and for that, data from literature and data sets will be used, as described next. First, the regions must be determined: region $1$ will be set as the State of São Paulo, which is responsible for $30\%$ of the national GDP, as reported by \textcite{ibge_produto_2024}. Other states, political regions or even groups of cities could be chosen, the only limited is that the smallest region in the model cannot be smaller than $0.03$ of the grand total, to guarantee stability of the model, as pointed by \textcite{konopkova_pitfalls_2019}.

The capital elasticity of production $\alpha$ used in the literature is $0.30$, and this value is used for region $2$. Region $1$ must be considered as a more capital intensive region, and for this reason the capital participation in production is higher, at least $30\%$, resulting in $\alpha_{1} = 0.36$. The intertemporal discount factor $\beta$; 
interest-rate smoothing parameter $\gamma_{R}$; 
interest-rate sensitivity in relation to inflation $\gamma_{\pi}$; 
interest-rate sensitivity in relation to product $\gamma_{Y}$; 
capital depreciation rate $\delta$; 
price stickness parameter $\theta$;
autoregressive parameter of productivity in region $1$ $\rho_{A1}$;
autoregressive parameter of productivity in region $2$ $\rho_{A2}$;
autoregressive parameter of monetary policy $\rho_{M}$;
relative risk aversion coefficient $\sigma$;
relative labor weight in utility $\phi$;
marginal disutility of labor supply $\varphi$;
elasticity of substitution between intermediate goods $\psi$;
standard deviation of productivity shock $\sigma_{A\eta}$;
standard deviation of monetary shock $\sigma_{M}$
 are given by the literature, as in \textcite{costa_junior_understanding_2016}.

The weight of region 1 in total production $\theta_{Y}$ is determined by the ratio of São Paulo's GDP and Brazilian's GPD, given by Table \eqref{table:gdp}. The productivity ratio between regions $\theta_{Z}$ represents how much region $2$ is less productive than region $1$. For that, we consider the GDP per hours worked as a mesure of productivity \cite{krugman_defining_1997}.

The weight of good $1$ in consumption composition of region $1$ $\omega_{11}$ and the weight of good $1$ in consumption composition of region $2$ $\omega_{21}$ are given by... @@@

\end{comment}

In this section, we outline the calibration process to replicate Brazilian economic characteristics, utilizing both literature and datasets. First, a region must be selected: in this study, region $1$ corresponds to the State of São Paulo, which contributes $30\%$ to the national GDP, as reported by \textcite{ibge_produto_2024}. While other regions or groupings are viable options, such as groups of States or Cities, the limitation is that the smallest region cannot be less than $0.03$ of the total, ensuring model stability, as emphasized by \textcite{konopkova_pitfalls_2019}.

To characterize region $1$ as more capital-intensive, the capital elasticity of production $\alpha_{1}$ is set at $0.36$, compared to the commonly used value of $0.30$, which is used for region $2$. Various parameters, such as the 
intertemporal discount factor $\beta$; 
interest-rate smoothing parameter $\gamma_{R}$; 
interest-rate sensitivity in relation to inflation $\gamma_{\pi}$; 
interest-rate sensitivity in relation to product $\gamma_{Y}$; 
capital depreciation rate $\delta$; 
price stickness parameter $\theta$;
autoregressive parameter of productivity in region $1$ $\rho_{A1}$;
autoregressive parameter of productivity in region $2$ $\rho_{A2}$;
autoregressive parameter of monetary policy $\rho_{M}$;
relative risk aversion coefficient $\sigma$;
relative labor weight in utility $\phi$;
marginal disutility of labor supply $\varphi$;
elasticity of substitution between intermediate goods $\psi$;
standard deviation of productivity shock $\sigma_{A\eta}$;
standard deviation of monetary shock $\sigma_{M}$
, are drawn from existing literature, as documented in \textcite{costa_junior_understanding_2016} and \textcite{pereira_rbc_2021}.

The weight of region $1$ in total production, $\theta_{Y}$, is determined by the São Paulo GDP to Brazil GDP ratio, as presented in Table \eqref{table:gdp}. Additionally, the productivity ratio $\theta_{Z}$ quantifies the relative productivity of region $2$ compared to region $1$, utilizing GDP per hours worked as a measure \cite{krugman_defining_1997}.

\newpage

% --------------------------------------------------
% REGIONAL GDP
% --------------------------------------------------

\subsubsection{Brazilian GDP in 2021}

\vspace*{0.3cm}

{\singlespacing
\begin{center}
	\begin{longtblr}[
		label = {table:gdp},
		caption = {Brazilian GDP in 2021},
		remark{Source} = {\textcite{ibge_produto_2024}}]
		{rowhead = 1,
			colspec = {Q[l]|Q[r]|Q[si={table-format=3.2,table-number-alignment=center},r,white]}}
		\hline[2pt]
		\textbf{State} & \textbf{GPD (R\$)} & \textbf{Participation (\%)} \\ \hline[2pt]
		São Paulo & 2.719.751.231 & 30,2 \\ \hline
		Rio de Janeiro & 949.300.770 & 10,5 \\ \hline
		Minas Gerais & 857.593.214 & 9,5 \\ \hline
		Rio Grande do Sul & 581.283.677 & 6,5 \\ \hline
		Paraná & 549.973.062 & 6,1 \\ \hline
		Santa Catarina & 428.570.889 & 4,8 \\ \hline
		Bahia & 352.617.852 & 3,9 \\ \hline
		Distrito Federal & 286.943.782 & 3,2 \\ \hline
		Goiás & 269.627.874 & 3,0 \\ \hline
		Pará & 262.904.979 & 2,9 \\ \hline
		Mato Grosso & 233.390.203 & 2,6 \\ \hline
		Pernambuco & 220.813.522 & 2,5 \\ \hline
		Ceará & 194.884.802 & 2,2 \\ \hline
		Espírito Santo & 186.336.505 & 2,1 \\ \hline
		Mato Grosso do Sul & 142.203.766 & 1,6 \\ \hline
		Amazonas & 131.531.038 & 1,5 \\ \hline
		Maranhão & 124.980.720 & 1,4 \\ \hline
		Rio Grande do Norte & 80.180.733 & 0,9 \\ \hline
		Paraíba & 77.470.331 & 0,9 \\ \hline
		Alagoas & 76.265.620 & 0,8 \\ \hline
		Piauí & 64.028.303 & 0,7 \\ \hline
		Rondônia & 58.170.096 & 0,6 \\ \hline
		Sergipe & 51.861.397 & 0,6 \\ \hline
		Tocantins & 51.780.764 & 0,6 \\ \hline
		Acre & 21.374.440 & 0,2 \\ \hline
		Amapá & 20.099.851 & 0,2 \\ \hline
		Roraima & 18.202.579 & 0,2 \\ \hline
		Brasil & 9.012.142.000 & 100,0 \\ \hline[2pt]
	\end{longtblr}	
\end{center}
} %end of \singlespacing

\newpage

% --------------------------------------------------
% PARAMETERS
% --------------------------------------------------

\subsubsection{Parameter Calibration}\label{sec:calibration}

\vspace*{0.5cm}

\begin{center}
\begin{longtblr}[
	label = {table:parameter-calibration},
	caption = {Parameter Calibration},
	remark{Sources} = {The Author and \textcite{costa_junior_understanding_2016}}]
	{rowhead = 1,
	colspec = {Q[c]|Q[l]|Q[si={table-format=3.2,table-number-alignment=center},c,white]}}
	\hline[2pt]
	\textbf{Parameter} & \textbf{Definition} & \textbf{Calibration} \\ \hline[2pt]
	%$\alpha$         & capital elasticity of production & $0.35$ \\ \hline
	$\alpha_1$       & capital elasticity of production in region 1 & $0.4$ \\ \hline
	$\alpha_2$       & capital elasticity of production in region 2 & $0.3$ \\ \hline
	$\beta$          & intertemporal discount factor & $0.985$ \\ \hline
	$\gamma_{R}$     & interest-rate smoothing parameter & $0.79$ \\ \hline
	$\gamma_{\pi}$   & interest-rate sensitivity in relation to inflation & $2.43$ \\ \hline
	$\gamma_{Y}$     & interest-rate sensitivity in relation to product & $0.16$ \\ \hline
	$\delta$         & capital depreciation rate & $0.025$ \\ \hline
	$\theta$         & price stickness parameter & $0.8$ \\ \hline
	%$\theta_{C11}$   & weight of good 1 in demand of region 1 & $0.4$ \\ \hline
	%$\theta_{C12}$   & weight of good 2 in demand of region 1 & $0.4$ \\ \hline
	%$\theta_{C21}$   & weight of good 1 in demand of region 2 & $0.4$ \\ \hline
	%$\theta_{C22}$   & weight of good 2 in demand of region 2 & $0.4$ \\ \hline
	%$\theta_{PY1}$   & weight of region 1 in gross domestic product & $0.3$ \\ \hline
	$\theta_{Y}$     & weight of region 1 in total production & $0.3$ \\ \hline
	%$\theta_{\pi}$   & weight of region 1 inflation in total inflation & $0.5$ \\ \hline
	%$\theta_{P}$     & price proportion between regions & $1$ \\ \hline
	$\theta_{Z}$     & productivity proportion between regions & $0.7$ \\ \hline
	$\rho_{A1}$      & autoregressive parameter of productivity in region 1 & $0.95$ \\ \hline
	$\rho_{A2}$      & autoregressive parameter of productivity in region 2 & $0.95$ \\ \hline
	$\rho_{M}$       & autoregressive parameter of monetary policy & $0.9$ \\ \hline
	$\sigma$         & relative risk aversion coefficient & $2$ \\ \hline
	$\phi$           & relative labor weight in utility & $1$ \\ \hline
	$\varphi$        & marginal disutility of labor supply & $1.5$ \\ \hline
	$\psi$           & elasticity of substitution between intermediate goods & $8$ \\ \hline
	$\sigma_{A\eta}$ & standard deviation of productivity shock & $0.01$ \\ \hline
	$\sigma_{M}$     & standard deviation of monetary shock & $0.01$ \\ \hline
	$\omega_{11}$    & weight of good 1 in consumption composition of region 1 & $0.7$ \\ \hline
	$\omega_{21}$    & weight of good 1 in consumption composition of region 2 & $0.4$ \\ \hline[2pt]
\end{longtblr}	
\end{center}

\newpage

% --------------------------------------------------
% VARIABLES AT THE STEADY STATE
% --------------------------------------------------

\subsubsection{Variables at Steady State}

\vspace*{0.5cm}

\begin{center}
\begin{longtblr}[
	label = {table:ss-values},
	caption = {Variables at Steady State},
	remark{Source} = {The Author.}]
	{rowhead = 1,
	 colspec = {Q[c]|Q[si = {table-format=3.2, table-number-alignment=center}, c, white]}}
		\hline[2pt]
		\textbf{Variable} & \textbf{Steady State Value} \\
		\hline[2pt]
		$\langle \begin{matrix} P_{1} & Z_{A1} \end{matrix} \rangle$ & $\vec{\bm{1}}$ \\ \hline
		$\langle \begin{matrix} P_{2} & Z_{A2} \end{matrix} \rangle$ & $\langle \begin{matrix} 1 & .7 \end{matrix} \rangle$ \\ \hline
		$\langle \begin{matrix} Z_{M} & \pi & \pi_{1} & \pi_{2} \end{matrix} \rangle$ & $\vec{\bm{1}}$ \\ \hline
		%$\langle \begin{matrix} \varepsilon_{A1} & \varepsilon_{A2} & \varepsilon_{M} \end{matrix} \rangle$ & $\vec{\bm{0}}$ \\ \hline
		$R_{}$    & $.0402$ \\ \hline
		$\langle \begin{matrix} \Lambda_{1} & \Lambda_{2} \end{matrix} \rangle$ & $\langle \begin{matrix} .875 & .875 \end{matrix} \rangle$ \\ \hline
		$\langle \begin{matrix} W_{1} & W_{2} \end{matrix} \rangle$ & $\langle \begin{matrix} 2.2208 & .8221 \end{matrix} \rangle$ \\ \hline
		$\langle \begin{matrix} a_{1} & a_{2} \end{matrix} \rangle$ & $\langle \begin{matrix} 4.3957 & 1.3514 \end{matrix} \rangle$ \\ \hline
		$\langle \begin{matrix} b_{1} & b_{2} \end{matrix} \rangle$ & $\langle \begin{matrix} .2175 & .1631 \end{matrix} \rangle$ \\ \hline
		$\langle \begin{matrix} Y_{1} & Y_{2} \end{matrix} \rangle$ & $\langle \begin{matrix} 2.6811 & 1.315 \end{matrix} \rangle$ \\ \hline
		$\langle \begin{matrix} I_{1} & I_{2} \end{matrix} \rangle$ & $\langle \begin{matrix} .5832 & .2145 \end{matrix} \rangle$ \\ \hline
		$\langle \begin{matrix} K_{1} & K_{2} \end{matrix} \rangle$ & $\langle \begin{matrix} 23.3263 & 8.5808 \end{matrix} \rangle$ \\ \hline
		$\langle \begin{matrix} C_{1} & C_{2} \end{matrix} \rangle$ & $\langle \begin{matrix} 2.0979 & 1.1005 \end{matrix} \rangle$ \\ \hline
		$\langle \begin{matrix} Q_{1} & Q_{2} \end{matrix} \rangle$ & $\langle \begin{matrix} 1.842 & 1.9601 \end{matrix} \rangle$ \\ \hline
		$\langle \begin{matrix} C_{11} & C_{12} \end{matrix} \rangle$ & $\langle \begin{matrix} 2.7051 & 1.1593 \end{matrix} \rangle$ \\ \hline
		$\langle \begin{matrix} C_{21} & C_{22} \end{matrix} \rangle$ & $\langle \begin{matrix} .8628 & 1.2943 \end{matrix} \rangle$ \\ \hline
		$\langle \begin{matrix} L_{1} & L_{2} \end{matrix} \rangle$ & $\langle \begin{matrix} .6338 & .9797 \end{matrix} \rangle$ \\ \hline[2pt]
	\end{longtblr}
	
\end{center}

% \newpage

%%%%%%%%%%%%%%%%%%%%%%%%%%%%%%%%%%%%%%%%%%%%%%%%%%%%%%%%

%% --------------------------------------------------
%% PARAMETER CALIBRATION
%% --------------------------------------------------
%
%\subsubsection{Parameter Calibration}
%
%\vspace*{-1cm}
%
%\begin{center}
%	
%	\begin{align}
	%		\begin{bmatrix}
		%			\phi       \\
		%			\varphi    \\
		%			\sigma     \\
		%			\beta      \\
		%			\delta     \\
		%			\psi       \\
		%			\theta     \\
		%			\alpha     \\
		%			\gamma_{R}   \\
		%			\gamma_{\pi} \\
		%			\gamma_{Y}   \\
		%			\rho_A     \\
		%			\rho_M     \\
		%			\theta_C   \\
		%			\theta_I
		%		\end{bmatrix} = 
	%		\begin{bmatrix}
		%			\phi       \\
		%			\varphi    \\
		%			\sigma     \\
		%			\beta      \\
		%			\delta     \\
		%			\psi       \\
		%			\theta     \\
		%			\alpha     \\
		%			\gamma_{R}   \\
		%			\gamma_{\pi} \\
		%			\gamma_{Y}   \\
		%			\rho_A     \\
		%			\rho_M     \\
		%			\theta_C   \\
		%			\theta_I   
		%		\end{bmatrix}
	%	\end{align}
%	
%\end{center}
%
%% --------------------------------------------------
%% VARIABLES AT THE STEADY STATE
%% --------------------------------------------------
%
%\subsubsection{Variables at the Steady State}
%
%\vspace*{-1cm}
%
%\begin{align}
%	\begin{bmatrix}
	%		P \\
	%		Z_A \\
	%		P^\ast \\
	%		\pi \\
	%		Z_M \\
	%		R \\
	%		\Lambda \\
	%		W \\
	%		Y \\
	%		C \\
	%		I \\
	%		K \\
	%		L
	%	\end{bmatrix} = 
%	\begin{bmatrix}
	%		P \\
	%		Z_A \\
	%		P^\ast \\
	%		\pi \\
	%		Z_M \\
	%		R \\
	%		\Lambda \\
	%		W \\
	%		Y \\
	%		C \\
	%		I \\
	%		K \\
	%		L
	%	\end{bmatrix}
%\end{align}
%
%\newpage	

\end{document}