% --------------------------------------------------
% DOCUMENT CLASS
% --------------------------------------------------

\documentclass[../thesis.tex]{subfiles}

\begin{document}
	
	\newpage
	
	\section*{\center Agradecimentos} %Acknowledgments
	
	\begin{spacing}{1.1}
		
	A dissertação é a reta final desta longa jornada chamada mestrado. Impossível chegar aqui sozinho: somente com o apoio da família, amigos, professores e instituições é que conseguimos dar este importante passo na vida acadêmica. Só tenho a agradecer:
	
	À CAPES, pele bolsa de estudos que permite aos alunos terem dedicação exclusiva durante o mestrado. Ao Banco do Brasil, pelo programa Pesquisador BB, um incentivo para os funcionários cursarem mestrado e doutorado. Ao PPGDE-UFPR, professores e servidores, pela excelência do ensino e administração do curso, em especial ao coordenador José Felipe Araujo de Almeida, à vice-coordenadora Kênia Barreiro de Souza e à Laís de Lima Rocha.

	Aos professores do PPGDE-UFPR, pela dedicação incondicional ao ensino, em especial aqueles que tive o privilégio de ser aluno: professores Armando Vaz Sampaio, Marcos Minoru Hasegawa, Maurício Vaz Lobo Bittencourt, Fernando Motta Correia, Victor Rodrigues de Oliveira, Kênia Barreiro de Souza, Terciane Sabadini Carvalho, Vinicius de Almeida Vale. % Aos professores que, mesmo longe, não hesitaram em responder os emails de um aluno desconhecido: professores Mario Solis-Garcia e Celso José Costa Junior. Aos professores que não poupam esforços em manter o aplicativo e o fórum do Dynare: Willi Mutschler e Johannes Pfeifer.
	
	Ao meu orientador Armando Vaz Sampaio, pela atenção, entusiasmo e sugestões que fizeram diferença na construção desta dissertação. Aos integrantes da banca de qualificação e de defesa, professores Celso José Costa Junior e Fernando Motta Correia, pelas importantes sugestões para melhoria e continuidade desta pesquisa.

	Ao meu chefe Greisson Almeida Pereira, por acreditar neste projeto desde a primeira versão e pelas importantes sugestões que enriqueceram a pesquisa. Ao meu amigo e novo chefe Julio César da Cunha Lopes, pela paciência e compreensão.
	
	Aos meus amigos do mestrado, em especial Felipe Duplat Luz e Matheus Antunio Canhete Da Silva, pela companhia e pelos questionamentos que enriqueceram este trabalho.
	
	Ao meu mentor Flávio von der Osten, por todos os conselhos e cobranças.
	
	Ao Professor João Ricardo Mendes Gonçalves Costa Filho, pela sua inestimável contribuição, a qual foi fundamental para o sucesso desta pesquisa.
	
	À minha heroína Kellen, pela confiança e pelo apoio incondicional.
	
	À minha mãe, pelo exemplo de força e perseverança.
	
	A todos que acompanharam esta jornada e me deram apoio e coragem, meu muito obrigado.
	
	\end{spacing}
	
	% no page number:
	\thispagestyle{empty}

\end{document}