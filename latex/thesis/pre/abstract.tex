% --------------------------------------------------
% DOCUMENT CLASS
% --------------------------------------------------

\documentclass[../thesis.tex]{subfiles}

\begin{document}

\newpage

% --------------------------------------------------
% RESUMO
% --------------------------------------------------

{\selectlanguage{brazilian}
	\onehalfspacing
	\begin{abstract}
		\vspace*{0.5cm}
		
		% O presente projeto de pesquisa propõe criar um modelo DSGE\footnote{(\textit{Dynamic and Stochastic General Equilibrium} ou Equilíbrio Geral Dinâmico e Estocástico)} regional para investigar os impactos da taxa de juros nominal sobre o produto interno bruto de uma região brasileira. Além dos elementos tradicionais da teoria Novo-Keynesiana, como competição monopolística e fricções de preços, o modelo apresenta duas regiões que se comunicam através do consumo do bem final de cada região por ambas. Estas regiões se diferenciam pelo nível de produtividade e pela participação do capital na função de produção da firma produtora de bens intermediários. As funções impulso-resposta demonstram que regiões com estruturas econômicas diferentes reagem de forma diferente a um choque de política monetária.
		
		Esta pesquisa tem como objetivo criar um modelo DSGE\footnote{\textit{Dynamic and Stochastic General Equilibrium} ou Equilíbrio Geral Dinâmico e Estocástico} regional para investigar os impactos da taxa de juros nominal sobre o produto interno bruto de uma região de um dado país. Além dos elementos tradicionais da teoria Novo-Keynesiana, como competição monopolística e fricções de preços, o modelo apresenta duas regiões que se comunicam através do consumo do bem final de cada região por ambas. Estas regiões se diferenciam pelo nível de produtividade e pela participação do capital na função de produção da firma produtora de bens intermediários. As funções impulso-resposta demonstram que regiões com estruturas econômicas diferentes reagem de forma diferente a um choque de política monetária. A região mais intensiva em capital é mais sensível ao choque de política monetária, como esperado.
		
	\end{abstract}
}

% no page number:
\thispagestyle{empty}

\newpage

% --------------------------------------------------
% ABSTRACT
% --------------------------------------------------

\pdfbookmark[1]{Abstract}{abstract}

{\onehalfspacing
	\begin{abstract}
	\vspace*{0.5cm}
		
	This research aims to create a regional DSGE\footnote{Dynamic and Stochastic General Equilibrium} model to investigate the impact of the nominal interest rate on the gross domestic product of a region of a given country. In addition to the traditional elements of New Keynesian theory, such as monopolistic competition and price frictions, the model features two regions that communicate through the consumption of the final good from each region by both. These regions differ in productivity levels and the share of capital in the production function of the intermediate-goods firm. The impulse-response functions demonstrate that regions with different economic structures react differently to a monetary policy shock. The more capital-intensive region is more sensitive to the monetary policy shock, as expected.
			
	% The present research project aims to develop a Dynamic and Stochastic General Equilibrium (DSGE) model to investigate the effects of the nominal interest rate on the Gross Domestic Product (GDP) of a Brazilian region.
			
	\end{abstract}
}
	
	% no page number:
	\thispagestyle{empty}
	
\end{document}