% --------------------------------------------------
% DOCUMENT CLASS
% --------------------------------------------------

\documentclass[
thesis.tex
]{subfiles}

\begin{document}

\newpage

\subsection{Definitions and Lemmas}

The objective of this appendix is to present the definitions and lemmas used throughout the text.

\subsubsection*{Household}

\begin{comment}
	
	% --------------------------------------------------
	% Household Maximization Problem
	% --------------------------------------------------
	
	\begin{definition}[Household Maximization Problem]
		{\singlespacing
			The utility function is:
			\begin{itemize}
				\item strictly increasing in consumption $C$;
				\item strictly increasing in leisure $l$;
				\item strictly concave;
				\item twice continuously differentiable;
				\item the composite consumption good $C$ is also the numeraire good, so that its price equals one: $p_C=1$;
				\item to avoid corner solutions, the Inada conditions\footnotemark{} hold. \footnotetext{see definition \ref{def:Inada Condition}.}
		\end{itemize}}
		
		Consider a representative household that maximizes an utility function $u$ that depends on consumption $C_t$ and labor $L_t$:
		\begin{align}
			u \equiv u \left( C_t, L_t \right)
		\end{align}
		
		The utility function is considered to be convex (when a variable increases, the respective marginal utility diminishes)\footnotemark{}: \footnotetext{Consider the following notation: given two variables $X$ and $Y$, the first and second partial derivatives are: $Y_X := \displaystyle\frac{\partial Y}{\partial X}$ and $Y_{XX} := \displaystyle\frac{\partial^2 Y}{\partial X^2}$.}
		\begin{align*}
			u_{C} > 0 \text{,}\quad u_{CC} < 0 \text{,}\quad
			u_{L} > 0 \text{,}\quad u_{LL} < 0
		\end{align*}
		
	\end{definition}
	
\end{comment}

\begin{definition}[Discount Factor $\beta$]
	
	Other things the same, a unit of consumption enjoyed tomorrow is less valuable (yields less utility) than a unit of consumption enjoyed today \cite[Lecture 2, p.1]{solis-garcia_ucb_2022}.
	
\end{definition}

% --------------------------------------------------
% Inada Condition
% --------------------------------------------------

\begin{definition}[Inada Condition] \label{def:Inada Condition}
	The Inada conditions \cite{inada_two-sector_1963} avoid corner solutions. For this purpose, it is assumed that the partial derivatives $u_C$ and $u_L$ of the function $u(C, L)$ satisfy the following rules:
	\begin{align}
		\lim_{C\to 0} u_C(C,L^\ast) = \infty \quad \text{and} \quad
		\lim_{C\to \infty} u_C(C,L^\ast) = 0 \\
		\lim_{L\to 0} u_C(C^\ast,L) = \infty \quad \text{and} \quad
		\lim_{L\to \infty} u_C(C^\ast,L) = 0 \nonumber
	\end{align}
	where $C^\ast,L^\ast \in \mathbb{R}_{++}$ and $u_j$ is the partial derivative of the utility function with respect to $j=C,L$ \cite[Lecture 1, p.2]{solis-garcia_ucb_2022}
\end{definition}

% --------------------------------------------------
% Transversality Condition
% --------------------------------------------------

\begin{definition}[Transversality Condition]
	\cite[Lecture 4, p.4]{solis-garcia_ucb_2022}
\end{definition}

% --------------------------------------------------
% FIRMS
% --------------------------------------------------

\subsubsection*{Firms}

% --------------------------------------------------
% MARGINAL COST
% --------------------------------------------------

\begin{lemma}[Marginal Cost]\label{lemma:marginal-cost}
	The Lagrangian multiplier $\Lambda_t$ is the nominal marginal cost of the intermediate-good firm:
	\begin{align}
		MC_t \coloneq \frac{\partial TC_t}{\partial Y_t} = \Lambda_t
	\end{align}
	
	\begin{proof}
		\textcite[p.449]{simon_mathematics_1994}.
	\end{proof}
	
\end{lemma}

% --------------------------------------------------
% Constant Returns to Scale
% --------------------------------------------------

\begin{definition}[Constant Returns to Scale]
	\cite[Lecture 1, p.5]{solis-garcia_ucb_2022}
\end{definition}

\begin{definition}[Homogeneous Function of Degree $k$]
	\cite[Lecture 1, p.5]{solis-garcia_ucb_2022}
\end{definition}

\subsubsection*{Monetary Authority}

\subsubsection*{Shocks}

\subsubsection*{Equilibrium Conditions}

\begin{definition}[Competitive Equilibrium]
	\cite[Lecture 1, p.6]{solis-garcia_ucb_2022}
\end{definition}

% --------------------------------------------------
% STEADY STATE
% --------------------------------------------------

\subsubsection*{Steady State}

% --------------------------------------------------
% INFLATION LEMMA
% --------------------------------------------------

\begin{lemma}[Steady State Inflation]\label{lemma:steady-state-inflation}
	
	In steady state, prices are stable $P_t = P_{t-1} = P$ and the gross inflation rate is one.
	\begin{proof} Equation \ref{eq:ss-gross-inflation-rate}. \end{proof}  \end{lemma}

\begin{corollary}\label{coro:steady-state-YKL}
	
	In steady state, all firms have the same level of production $Y$ and therefore demand the same amount of factors, capital $K$ and labor $L$.
	\begin{align*}
		P_t = P_{t-1} = P \implies 
		\begin{pmatrix}
			Y_j & K_j & L_j
		\end{pmatrix} =
		\begin{pmatrix}
			Y & K & L
		\end{pmatrix}
	\end{align*}
	
\end{corollary}

% --------------------------------------------------
% LOG-LINEARIZATION
% --------------------------------------------------

\subsubsection*{Log-linearization}

% --------------------------------------------------
% PERCENTAGE DEVIATION
% --------------------------------------------------

\begin{definition}[PERCENTAGE DEVIATION]\label{def:percentage-deviation}
	
	The percentage deviation of a variable $x_t$ from its steady state is given by \cite[Lecture 6, p.2]{solis-garcia_ucb_2022}:
	\begin{align}
		\hat{x}_t \coloneq \frac{x_t - x}{x} \label{eq:percentage-deviation}
	\end{align}
	
\end{definition}

% --------------------------------------------------
% UHLIG'S RULES
% --------------------------------------------------

\begin{lemma}[UHLIG'S RULES]\label{lemma:uhligs-rules}
	
	The Uhlig's rules are a set of approximations used to log-linearize equations \cite[Lecture 6, p.2]{solis-garcia_ucb_2022}.
	
	\begin{itemize}
		\item Rule 1: \label{uhlig-rule-1}
		
		\( x_t = x(1 + \hat{x}_t) \) 
		
		\item Rule 2 (Product):
		
		
		
		\item Rule 3 (Exponential):
		
		
	\end{itemize}
	
\end{lemma}

\begin{corollary}[Logarithm Rule]\label{coro:logarithm-rule}
	
	\begin{align*}
		\ln x_t \approx \ln x + \hat{x}_t
	\end{align*}
	
\end{corollary}

% --------------------------------------------------
% LEVEL DEVIATION
% --------------------------------------------------

\begin{definition}[LEVEL DEVIATION]\label{def:level-deviation}
	
	The level deviation of a variable $u_t$ from its steady state is given by: \cite[Lecture 9, p.9]{solis-garcia_ucb_2022}
	\begin{align}
		\widetilde{u}_t \coloneq u_t - u \label{eq:level-deviation}
	\end{align}
	
\end{definition}

% --------------------------------------------------
% UHLIG'S RULES FOR LEVEL DEVIATIONS
% --------------------------------------------------

\begin{lemma}[UHLIG'S RULES FOR LEVEL DEVIATIONS]\label{lemma:level-rules}
	
	Uhlig's rules can be applied to level deviations in order to log-linearize equations \cite[Lecture 9, p.9]{solis-garcia_ucb_2022}.
	
	\begin{itemize}
		\item Rule 1:
		\begin{align}
			\label{lemma:level-rule-1a}
			u_t &= u + \widetilde{u}_t \\
			\label{lemma:level-rule-1b}
			u_t &= u\left(1+ \frac{\widetilde{u}_t}{u} \right)
		\end{align}
		
		\item Rule 2 (Product):
		
		\item Rule 3 (Exponential):
		
		\item Rule 4 (Logarithm):
		
		\item Rule 5 (Percentage and Level Deviations)
		
	\end{itemize}
	
\end{lemma}

% --------------------------------------------------
% PRODUCT OPERATOR
% --------------------------------------------------

\begin{lemma}[LEVEL DEVIATION OF THE PRESENT VALUE DISCOUNT FACTOR]\label{product-operator}
	
	The level deviation of the present value discount factor is equivalent to \cite[Lecture 13, p.6]{solis-garcia_ucb_2022}:
	\begin{align}
		\label{eq:product-operator}
		\prod_{k=0}^{s-1}(1+R_{t+k}) = (1 + R)^s \left( 1 + \frac{1}{1 + R} \sum_{k=0}^{s-1} \widetilde{R}_{t+k} \right)
	\end{align}
	
	\begin{proof}
		Substitute the interest rate by the gross interest rate $GR_t = 1 + R_t$ and apply rule \ref{lemma:level-rule-1b}:
		\begin{align*}
			& \prod_{k=0}^{s-1}(1+R_{t+k}) = \prod_{k=0}^{s-1}(GR_{t+k})
			&\implies \nonumber \\
			& GR \times \dots \times GR \left( 1 + \frac{1}{GR} \widetilde{GR}_t + \frac{1}{GR} \widetilde{GR}_{t+1} + \dots + \frac{1}{GR} \widetilde{GR}_{t+s-1} \right)
			&\implies \nonumber \\
			& GR^s \left( 1 + \frac{1}{GR} \sum_{k=0}^{s-1} \widetilde{GR}_{t+k} \right)
			&\implies \nonumber \\
			& (1 + R)^s \left( 1 + \frac{1}{1 + R} \sum_{k=0}^{s-1} \widetilde{R}_{t+k} \right) &\,
		\end{align*}
	\end{proof}
	
\end{lemma}

% --------------------------------------------------
% PERCENTAGE DIVISION
% --------------------------------------------------

%\begin{lemma}[PERCENTAGE DIVISION]\label{lemma:percentage-division}
%	The division of gross percentages is equivalent to the subtraction of percentages.
%	\begin{align}
	%		\frac{1+x}{1+y} \approx 1 + x - y
	%	\end{align}
%	where \( x,y \in [0,1] \) and \( X,Y \geq 0 \).
%	\begin{proof}
	%		\begin{align}
		%			\frac{1+x}{1+y} = \frac{1 + \frac{X}{100}}{1 + \frac{Y}{100}} = \frac{100 + X}{100} \cdot \frac{100}{100 + Y} = \frac{100 + X}{100 + Y}
		%		\end{align}
	%	\end{proof}
%\end{lemma}

% --------------------------------------------------
% GEOMETRIC SERIES
% --------------------------------------------------

\begin{definition}[Geometric Series]\label{def:geometric-series}
	
	A geometric series is the sum of the terms of a geometric sequence.
	\begin{align*}
		S_\infty = \sum_{i=0}^{\infty} ar^i \implies 
		S_\infty = \frac{a}{1-r} \; , \; |r| <1
	\end{align*}
	
\end{definition}

% --------------------------------------------------
% LAG OPERATOR
% --------------------------------------------------

\begin{definition}[LAG AND LEAD OPERATORS]\label{def:lag-operator}
	The lag operator $\mathbb{L}$ is a mathematical operator that represents the backshift or lag of a time series \cite[Lecture 13, p.9]{solis-garcia_ucb_2022}:
	\begin{align*}
		\mathbb{L} x_t            & = x_{t-1}              \\
		(1 + a\mathbb{L})y_{t+2} & = y_{t+2} + ay_{t+1}
	\end{align*}
	
	Analogously, the lead operator $\mathbb{L}^{-1}$ (or inverse lag operator) yields a variable's lead \cite[Lecture 13, p.9]{solis-garcia_ucb_2022}:
	\begin{align*}
		\mathbb{L}^{-1} x_t            & = x_{t+1}              \\
		(1 + a\mathbb{L}^{-1}) y_{t+2} & = y_{t+2} + ay_{t+3}
	\end{align*}
\end{definition}

\subsubsection*{Canonical NK Model}

% --------------------------------------------------
% DEFINITION
% --------------------------------------------------

\begin{comment}
	
	\begin{definition}[Canonical NK Model]
		
		\cite[Lecture 13, p.7]{solis-garcia_ucb_2022}
		
		3.1.2 Back to the pricing equation:
		
		log-linearize the left hand equation:
		\begin{align*}
			&\mathbb{E}_t \sum_{s=0}^{\infty} 
			\left[ 
			\left( \frac{\theta}{1+R} \right)^s
			\left( \frac{P_t^\ast Y_{t+s}(j)}{1 + \frac{1}{1+R}
				\sum_{k=0}^{s-1} \widetilde{R}_{t+k}} \right) 
			\right]
			\implies \\
			&\mathbb{E}_t \sum_{s=0}^{\infty} 
			\left[ 
			\left( \frac{\theta}{1+R} \right)^s
			\left( \frac{P^\ast Y(j)(1+\widehat{P}_t^\ast + \widehat{Y}_{t+s}(j))}{1 + \frac{1}{1+R}
				\sum_{k=0}^{s-1} \widetilde{R}_{t+k}} \right) 
			\right] \implies \\
			&\mathbb{E}_t \sum_{s=0}^{\infty} 
			\left[ 
			\left( \frac{\theta}{1+R} \right)^s
			\left( \frac{P_t^\ast Y_{t+s}(j)}{\frac{(1+R)+\sum_{k=0}^{s-1} \widetilde{R}_{t+k}}{1+R}} \right) 
			\right]
			\implies \\
			&\mathbb{E}_t \sum_{s=0}^{\infty} 
			\left[ 
			\left( \frac{\theta}{1+R} \right)^s
			\left( \frac{P_t^\ast Y_{t+s}(j)(1+R)}{(1+R)+\sum_{k=0}^{s-1} \widetilde{R}_{t+k}} \right) 
			\right]
			\implies \\
			&\mathbb{E}_t \sum_{s=0}^{\infty} 
			\left[ 
			\left( \frac{\theta}{1+R} \right)^s
			\left( \frac{P^\ast Y(j)(1+\widehat{P}_t^\ast + \widehat{Y}_{t+s}(j))(1+R)}{(1+R)+\sum_{k=0}^{s-1} \widetilde{R}_{t+k}} \right) 
			\right]
		\end{align*}
		
	\end{definition}
	
\end{comment}

\begin{definition}[Medium Scale DSGE Model]
	A Medium Scale DSGE Model has habit formation, capital accumulation, indexation, etc. \cite[p.208]{gali_monetary_2015}. 
	
	See Galí, Smets, and Wouters (2012) for an analysis of the sources of unemployment fluctuations in an estimated medium-scale version of the present model.
\end{definition}

\begin{definition}[Stochastic Process]
	\cite[Lecture 5, p.3]{solis-garcia_ucb_2022}.
\end{definition}

\begin{definition}[Markov Process]
	\cite[Lecture 5, p.4]{solis-garcia_ucb_2022}.
\end{definition}

\begin{definition}[first-order autoregressive process $AR(1)$]
	the first-order autoregressive process $AR(1)$ \cite[Lecture 5, p.4]{solis-garcia_ucb_2022}.
\end{definition}

\begin{definition}[Blanchard-Kahn Conditions]
	\cite[Hands on 5, p.14]{solis-garcia_ucb_2022}.
\end{definition}


% --------------------------------------------------
% DEFINITION
% --------------------------------------------------

% --------------------------------------------------
% DEFINITION
% --------------------------------------------------

% --------------------------------------------------
% DEFINITION
% --------------------------------------------------

% --------------------------------------------------
% DEFINITION
% --------------------------------------------------

% --------------------------------------------------
% DEFINITION
% --------------------------------------------------

% --------------------------------------------------
% DEFINITION
% --------------------------------------------------

% --------------------------------------------------
% DEFINITION
% --------------------------------------------------

% --------------------------------------------------
% DEFINITION
% --------------------------------------------------

% --------------------------------------------------
% DEFINITION
% --------------------------------------------------

% --------------------------------------------------
% DEFINITION
% --------------------------------------------------

% --------------------------------------------------
% DEFINITION
% --------------------------------------------------

% --------------------------------------------------
% DEFINITION
% --------------------------------------------------

% --------------------------------------------------
% DEFINITION
% --------------------------------------------------

% --------------------------------------------------
% DEFINITION
% --------------------------------------------------

% --------------------------------------------------
% DEFINITION
% --------------------------------------------------

\end{document}