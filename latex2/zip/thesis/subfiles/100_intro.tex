% --------------------------------------------------
% DOCUMENT CLASS
% --------------------------------------------------

\documentclass[
	thesis.tex
	]{subfiles}

\begin{document}

	The importance of macroeconomic modeling as a tool for studying the connections between monetary economy and the outcomes of a country's aggregates is undeniable, as stated by \textcite{gali_monetary_2015}. Considering as well that Brazilian regions possess heterogeneous economic matrices and sectors that respond in different ways to monetary authority decisions, as indicated by \textcite{bertanha_efeitos_2008}, the need for a structural model capable of relating macroeconomic variables to regional variables becomes evident.
	
	In this context, the present research project proposes the development of a macroeconomic model with regional extensions, using the DSGE methodology\footnote{Dynamic and Stochastic General Equilibrium}, which can demonstrate the existing relationships among the various considered variables and present impulse response functions that illustrate these relationships. With this model, we aim to investigate the existing relationship between the nominal interest rate of the Brazilian economy and the level of regional gross domestic product.
	
	\subsection*{Problem and Justification}
	
	The main issue to be investigated is the impacts of monetary authority decisions --- especially changes in the nominal interest rate --- on regional macroeconomic variables, particularly the Gross Domestic Product (GDP) of a given Brazilian region (such as a state, for example).
	
	Given that Brazilian regions have distinct economic matrices (agriculture, industry, extraction, etc.), and within each of these specializations, some sectors are more labor-intensive while others are capital-intensive, it is plausible to assume that regional diversity allows each region to react differently to changes in the interest rate.
	
	Given the problem, we need to determine how the study will be conducted. As this is a topic that combines knowledge from Macroeconomics and Regional Economics, it will be necessary to address the main concepts from both areas to then determine a methodology capable of integrating all of this content.
	
	Regional Economics investigations borrow tools from Macroeconomics, as pointed out by \textcite{rickman_modern_2010}: the Leontief input-output model, the Walrasian general equilibrium applied model, and the system of macroeconometric equations are some examples of how models from one field have been adapted and used by the other. In this sense, the proposal for this work is to use a structural model (which is a tool from Macroeconomics) to determine the relationships between a macro variable and regional variables, and subsequently use Brazilian economic data to determine the level of correlation between them.
	
	Numerous studies address the effects of national aggregates on regional variables, and these will be appropriately presented in section \ref{sec:literature-review}. However, in these studies, we have not found one that specifically investigates the relationship between the national nominal interest rate and regional GDP.
	
	The significance of this work can be identified by recognizing that, given the diversity of Brazilian regions, it is not plausible that a single macroeconomic variable will have the same effect in each of them (or at least not with the same intensity). Thus, a tool capable of quantifying the regional effect of a macroeconomic variable is an important addition to economic literature, as it investigates the transmission mechanisms of monetary policy to the regional aggregate. Additionally, it also adds to the array of policy evaluation instruments, such that various economic agents can use this tool to determine the conduct of their own internal policies. For example, banks can quantify the credit interest rate for a specific region based on the projected interest rate, considering the needs and potential development of each region separately from the rest of the country.

	\subsection*{Objectives}
	
	The main objective is to create a structural model capable of relating a macroeconomic variable (the nominal interest rate) to a regional variable (the Gross Domestic Product of a Brazilian region), in order to assess the impact of an expansionary (or contractionary) monetary policy on a specific Brazilian region and the magnitude of that impact.
	
	The specific objectives are 
		\begin{enumerate*}[label=(\arabic*)]
			\item elaborate a NK DSGE model with households, firms, monetary authority, price stickiness, productivity and monetary shocks to demonstrate that the nominal interest rate determined by the monetary authority influences the national GDP; 
			\item determine which variables must be regionalized in order to make a regional environment in order to demonstrate that two regions may have different responses to the monetary policy shocks; 
			\item produce IRF and analyse the results of both models.
		\end{enumerate*}
	
\end{document}